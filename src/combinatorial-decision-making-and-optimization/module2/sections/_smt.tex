\chapter{Satisfiability modulo theory (SMT)}


\section{First-order logic for SMT}


\subsection{Syntax}

\begin{remark}
    Only quantifier-free fragments will be considered in this course.
\end{remark}

\begin{description}
    \item[Functions] \marginnote{Functions}
        The set of all the functions is denoted as $\Sigma^F = \bigcup_{k \geq 0} \Sigma^F_k$
        where $\Sigma^F_k$ denotes the set of $k$-ary functions.

        \begin{description}
            \item[Constants] $\Sigma^F_0$
        \end{description}

    \item[Predicates] \marginnote{Predicates}
        The set of all the predicates is denoted as $\Sigma^P = \bigcup_{k \geq 0} \Sigma^P_k$
        where $\Sigma^P_k$ denotes the set of $k$-ary predicates.

        \begin{description}
            \item[Propositional symbols] $\Sigma^P_0$
        \end{description}

    \item[Signature] \marginnote{Signature}
        The set of the non-logical symbols of FOL is denoted as:
        \[ \Sigma = \Sigma^F \cup \Sigma^P \]

    \item[Terms] \marginnote{Terms}
        The set of terms over $\Sigma$ is denoted as $\mathbb{T}^\Sigma$.

    \item[Formulas] \marginnote{Formulas}
        The set of formulas over $\Sigma$ is denoted as $\mathbb{F}^\Sigma$.
\end{description}


\subsection{Semantics}

\begin{description}
    \item[$\mathbf{\Sigma}$-model] \marginnote{$\Sigma$-model}
        Pair $\mathcal{M} = \langle M, (\cdot)^\mathcal{M} \rangle$ defined on a given $\Sigma$ where:
        \begin{itemize}
            \item $M$ is the universe of $\mathcal{M}$.
            \item $(\cdot)^\mathcal{M}$ is a mapping such that:
            \begin{itemize}
                \item $\forall f \in \Sigma^F_k: f^\mathcal{M} \in \{ \varphi \mid \varphi: M^k \rightarrow M \}$.
                \item $\forall p \in \Sigma^P_k: p^\mathcal{M} \in \{ \varphi \mid \varphi: M^k \rightarrow \{ \texttt{true}, \texttt{false} \} \}$.
            \end{itemize}
        \end{itemize} 

    \item[Interpretation] \marginnote{Interpretation}
        Extension of the mapping function $(\cdot)^\mathcal{M}$ to terms and formulas:
        \begin{itemize}
            \item $\top^\mathcal{M} = \texttt{true}$ and $\bot^\mathcal{M} = \texttt{false}$.
            \item $(f(t_1, \dots, t_k))^\mathcal{M} = f^\mathcal{M}(t_1^\mathcal{M}, \dots, t_k^\mathcal{M})$ and 
                $(p(t_1, \dots, t_k))^\mathcal{M} = p^\mathcal{M}(t_1^\mathcal{M}, \dots, t_k^\mathcal{M})$.
            \item $\texttt{ite}(\varphi, t_1, t_2)^\mathcal{M} = \begin{cases}
                t_1^\mathcal{M} & \text{if $\varphi^\mathcal{M} = \texttt{true}$} \\
                t_2^\mathcal{M} & \text{if $\varphi^\mathcal{M} = \texttt{false}$}
            \end{cases}$
        \end{itemize}

        \begin{remark}
            \texttt{ite} is an auxiliary function to capture the if-else construct.
        \end{remark}
\end{description}


\subsection{$\mathbf{\Sigma}$-theory}

\begin{description}
    \item[Satisfiability] \marginnote{Satisfiability}
        A model $\mathcal{M}$ satisfies a formula $\varphi \in \mathcal{F}^\Sigma$ if $\varphi^\mathcal{M} = \texttt{true}$.

    \item[$\mathbf{\Sigma}$-theory] \marginnote{$\Sigma$-theory}
        Possibly infinite set $\mathcal{T}$ of $\Sigma$-models.

    \item[$\mathbf{\mathcal{T}}$-satisfiability] \marginnote{$\mathcal{T}$-satisfiability}
        A formula $\varphi \in \mathbb{F}^\Sigma$ is $\mathcal{T}$-satisfiable if there exists a model $\mathcal{M} \in \mathcal{T}$ that satisfies it.

    \item[$\mathbf{\mathcal{T}}$-consistency] \marginnote{$\mathcal{T}$-consistency}
        A set of formulas $\{ \varphi_1, \dots, \varphi_k \} \subseteq \mathbb{F}^\Sigma$ is $\mathcal{T}$-consistent iff 
        $\varphi_1 \land \dots \land \varphi_k$ is $\mathcal{T}$-satisfiable.

    \item[$\mathbf{\mathcal{T}}$-entailment] \marginnote{$\mathcal{T}$-entailment}
        A set of formulas $\Gamma \subseteq \mathbb{F}^\Sigma$ $\mathcal{T}$-entails a formula $\varphi \in \mathbb{F}^\Sigma$ ($\Gamma \models_\mathcal{T} \varphi$) iff
        in every model $\mathcal{M} \in \mathcal{T}$ that satisfies $\Gamma$, $\varphi$ is also satisfied.

        \begin{remark}
            $\Gamma$ is $\mathcal{T}$-consistent iff $\Gamma \cancel{\models_\mathcal{T}} \bot$.
        \end{remark}

    \item[$\mathbf{\mathcal{T}}$-validity] \marginnote{$\mathcal{T}$-validity}
        A formula $\varphi \in \mathbb{F}^\Sigma$ is $\mathcal{T}$-valid iff $\varnothing \models_\mathcal{T} \varphi$.

        \begin{remark}
            $\varphi$ is $\mathcal{T}$-consistent iff $\lnot\varphi$ is not $\mathcal{T}$-valid.
        \end{remark}

        \begin{description}
            \item[Theory lemma] \marginnote{Theory lemma}
                $\mathcal{T}$-valid clause $c = l_1 \vee \dots \vee l_k$.
        \end{description}

    \item[$\mathbf{\mathcal{T}}$-expansion] \marginnote{$\mathcal{T}$-expansion}
        Given a $\Sigma$-model $\mathcal{M} = \langle M, (\cdot)^\mathcal{M} \rangle$ and $\Sigma' \supseteq \Sigma$,
        an expansion $\mathcal{M}' = \langle M', (\cdot)^{\mathcal{M}'} \rangle$ to $\Sigma'$ is any $\Sigma'$-model such that:
        \begin{itemize}
            \item $M' = M$.
            \item $\forall s \in \Sigma: s^{\mathcal{M}'} = s^\mathcal{M}$
        \end{itemize}

        \begin{remark}
            Given a $\Sigma$-theory $\mathcal{T}$, we implicitly consider the theory $\mathcal{T}'$ as:
            \[ \mathcal{T}' = \{ \mathcal{M}' \mid \mathcal{M}' \text{ is an expansion of a $\Sigma$-model } \mathcal{M} \} \]
        \end{remark}

    \item[Ground $\mathbf{\mathcal{T}}$-satisfiability] \marginnote{Ground $\mathcal{T}$-satisfiability}
        Given a $\Sigma$-theory $\mathcal{T}$, determine if a ground formula is $\mathcal{T}$-satisfiable over a $\Sigma$-expansion $\mathcal{T}'$.

    \item[Axiomatically defined theory] \marginnote{Axiomatically defined theory}
        Given a minimal set of formulas (axioms) $\Lambda \subseteq \mathbb{F}^\Sigma$,
        its corresponding theory is the set of all the models of $\Lambda$.
\end{description}

\begin{example}
    Let $\Sigma$ be defined as:
    \[ \Sigma^F_0 = \{ a, b, c, d \} \hspace{2em} \Sigma^F_1 = \{ f, g \} \hspace{2em} \Sigma^P_2 = \{ p \} \]
    A $\Sigma$-model $\mathcal{M} = \langle [0, 2\pi[, (\cdot)^\mathcal{M} \rangle$ can be defined as follows:
    \[ a^\mathcal{M} = 0 \hspace{2em} b^\mathcal{M} = \frac{\pi}{2} \hspace{2em} c^\mathcal{M} = \pi \hspace{2em} d^\mathcal{M} = \frac{3\pi}{2} \]
    \[ f^\mathcal{M} = \sin \hspace{2em} g^\mathcal{M} = \cos \hspace{2em} p^\mathcal{M}(x, y) \iff x > y \]

    To determine if $p(g(x), f(d))$ is $\mathcal{M}$-satisfiable, we have to expand $\mathcal{M}$.
    Let $\Sigma' = \Sigma \cup \{ x \}$. The expansion $\mathcal{M}'$ such that $x^{\mathcal{M}'} = \frac{\pi}{2}$ makes the formula satisfiable.
\end{example}


% \begin{description}
%     \item[Satisfiability modulo theory (SMT)] \marginnote{Satisfiability modulo theory (SMT)}
%         Satisfiability of a formula with respect to some background theory.

%         SMT extends SAT and exploits domain-specific reasoning (possibly with infinite domains).
% \end{description}