\chapter{Convolutional neural networks}


\begin{description}
    \item[Convolution neuron] \marginnote{Convolution neuron}
        Neuron influenced by only a subset of neurons in the previous layer.
        
    \item[Receptive field] \marginnote{Receptive field}
        Dimension of the input image influencing a neuron.

    \item[Convolutional layer] \marginnote{Convolutional layer}
        Layer composed of convolutional neurons.
        Neurons in the same convolutional layer share the same weights and work as a convolutional filter.

        \begin{remark}
            The weights of the filters are learned.
        \end{remark}

        A convolutional layer has the following parameters:
        \begin{descriptionlist}
            \item[Kernel size] \marginnote{Kernel size}
                Dimension (i.e. width and height) of the filter.

            \item[Stride] \marginnote{Stride}
                Offset between each filter application (i.e. stride $>1$ reduces the size of the output image).

            \item[Padding] \marginnote{Padding}
                Artificial enlargement of the image.
                
                In practice, there are two modes of padding:
                \begin{descriptionlist}
                    \item[Valid] No padding applied.
                    \item[Same] Apply the minimum padding needed.
                \end{descriptionlist}

            \item[Depth] \marginnote{Depth}
                Number of different kernels to apply (i.e. augment the number of channels in the output image).
        \end{descriptionlist}

        The dimension along each axis of the output image is given by:
        \[ \frac{W + P - K}{S} + 1 \]
        where:
        \begin{itemize}
            \item $W$ is the size of the image (width or height).
            \item $P$ is the padding.
            \item $K$ is the kernel size.
            \item $S$ is the stride.
        \end{itemize}

        \begin{remark}
            If not specified, a kernel is applied to all the channels of the input image in parallel (but the weights of the kernel change at each channel).
        \end{remark}
\end{description}


\begin{description}
    \item[Pooling]
        Layer that applies a function as a filter.

        \begin{descriptionlist}
            \item[Max-pooling] \marginnote{Max-pooling}
                Filter that computes the maximum of the pixels within the kernel.

            \item[Mean-pooling] \marginnote{Mean-pooling}
                Filter that computes the average of the pixels within the kernel.
        \end{descriptionlist}
\end{description}


\section{Parameters}

The number of parameters of a layer is given by:
\[ (K_\text{w} \cdot K_\text{h}) \cdot D_\text{in} \cdot D_\text{out} + D_\text{out} \]
where:
\begin{itemize}
    \item $K_\text{w}$ is the width of the kernel.
    \item $K_\text{h}$ is the height of the kernel.
    \item $D_\text{in}$ is the input depth.
    \item $D_\text{out}$ is the output depth.
\end{itemize}

Therefore, the number of FLOPS is of order:
\[ (K_\text{w} \cdot K_\text{h}) \cdot D_\text{in} \cdot D_\text{out} \cdot (O_\text{w} \cdot O_\text{h}) \]
where:
\begin{itemize}
    \item $O_\text{w}$ is the width of the output image.
    \item $O_\text{h}$ is the height of the output image.
\end{itemize}