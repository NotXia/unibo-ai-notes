\chapter{Automated planning definitions}

\begin{description}
    \item[Automated planning] \marginnote{Automated planning} 
        Given:
        \begin{itemize}
            \item An initial state.
            \item A set of actions an agent can perform (operators).
            \item The goal to achieve.
        \end{itemize}
        Automated planning finds a partially or totally ordered set of actions
        that leads an agent from the initial state to the goal.

    \item[Domain theory] \marginnote{Domain theory}
        Formal description of the executable actions.
        Each action has a name, pre-conditions and post-conditions.
        \begin{descriptionlist}
            \item[Pre-conditions] 
                Conditions that must hold for the action to be executable.
            \item[Post-conditions] 
                Effects of the action.
        \end{descriptionlist}

    \item[Planner] \marginnote{Planner}
        Process to decide the actions that solve a planning problem.
        In this phase, actions are considered:
        \begin{description}
            \item[Non decomposable] 
                An action is atomic (it starts and finishes).
                Actions interact with each other by reaching sub-goals.
            \item[Reversible] 
                Choices are backtrackable.
        \end{description}

        A planner can have the following properties:
        \begin{descriptionlist}
            \item[Correctness] \marginnote{Correct planner}
                The planner always finds a solution that leads from the initial state to the goal.
            \item[Completeness] \marginnote{Complete planner}
                The planner always finds a plan when it exits (planning is semi-decidable).
        \end{descriptionlist}

    \item[Execution] \marginnote{Execution}
        The execution is the implementation of a plan. 
        In this phase, actions are:
        \begin{descriptionlist}
            \item[Irreversible]
                An action that has been executed cannot (usually) be backtracked.
            \item[Non deterministic] 
                An action applied to the real world may have unexpected effects due to uncertainty.
        \end{descriptionlist}
\end{description}
