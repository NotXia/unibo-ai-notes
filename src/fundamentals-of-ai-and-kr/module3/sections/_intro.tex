\chapter{Introduction}


\section{Uncertainty}
\begin{description}
    \item[Uncertainty] \marginnote{Uncertainty}
        A task is uncertain if we have:
        \begin{itemize}
            \item Partial observations
            \item Noisy or wrong information
            \item Uncertain action outcomes
            \item Complex models
        \end{itemize}

        A purely logic approach leads to:
        \begin{itemize}
            \item Risks falsehood: unreasonable conclusion when applied in practice.
            \item Weak decisions: too many conditions required to make a conclusion.
        \end{itemize}
\end{description}


\subsection{Handling uncertainty}
\begin{descriptionlist}
    \item[Default/nonmonotonic logic] \marginnote{Default/nonmonotonic logic}
        Works on assumptions.
        An assumption can be contradicted by an evidence.

    \item[Rule-based systems with fudge factors] \marginnote{Rule-based systems with fudge factors}
        Formulated as premise $\rightarrow_\text{prob.}$ effect.
        Have the following issues:
        \begin{itemize}
            \item Locality: how can the probability account all the evidence.
            \item Combination: chaining of unrelated concepts.
        \end{itemize}

    \item[Probability] \marginnote{Probability}
        Assign a probability given the available known evidence.

        Note: fuzzy logic handles the degree of truth and not the uncertainty.
\end{descriptionlist}

\begin{description}
    \item[Decision theory] \marginnote{Decision theory}
        Defined as:
        \[ \text{Decision theory} = \text{Utility theory} + \text{Probability theory} \]
        where the utility theory depends on one's preferences.
\end{description}


\subsection{Probability}

\begin{description}
    \item[Sample space] \marginnote{Sample space}
        Set $\Omega$ of all possible worlds.
        \begin{descriptionlist}
            \item[Event] \marginnote{Event}
                Subset $A \subseteq \Omega$.
            \item[Sample point/Possible world/Atomic event] \marginnote{Sample point}
                Element $\omega \in \Omega$.
        \end{descriptionlist}

    \item[Probability space] \marginnote{Probability space}
        A probability space/model is a function $\prob{\cdot}: \Omega \rightarrow [0, 1]$ assigned to a sample space such that:
        \begin{itemize}
            \item $0 \leq \prob{\omega} \leq 1$
            \item $\sum_{\omega \in \Omega} \prob{\omega} = 1$
            \item $\prob{A} = \sum_{\omega \in A} \prob{\omega}$
        \end{itemize}

    \item[Random variable] \marginnote{Random variable}
        A function from an event to some range (e.g. reals, booleans, \dots).

    \item[Probability distribution] \marginnote{Probability distribution}
        For any random variable $X$:
        \[ \prob{X = x_i} = \sum_{\omega \text{ st } X(\omega)=x_i} \prob{\omega} \]

    \item[Proposition] \marginnote{Proposition}
        Event where a random variable has a certain value.
        \[ a = \{ \omega \,\vert\, A(\omega) = \texttt{true} \} \]
        \[ \lnot  a = \{ \omega \,\vert\, A(\omega) = \texttt{false} \} \]
        \[ (\texttt{Weather} = \texttt{rain}) = \{ \omega \,\vert\, B(\omega) = \texttt{rain} \} \]

    \item[Prior probability] \marginnote{Prior probability}
        Prior/unconditional probability of a proposition based on known evidence.
        
    \item[Probability distribution (all)] \marginnote{Probability distribution (all)}
        Gives all the probabilities of a random variable.
        \[ \textbf{P}(A) = \langle \prob{A=a_1}, \dots, \prob{A=a_n} \rangle \]
    
    \item[Joint probability distribution] \marginnote{Joint probability distribution}
        The joint probability distribution of a set of random variables gives 
        the probability of all the different combinations of their atomic events.

        Note: Every question on a domain can, in theory, be answered using the joint distribution.
        In practice, it is hard to apply.

        \begin{example}
            $\textbf{P}(\texttt{Weather}, \texttt{Cavity}) = $
            \begin{center}
                \small
                \begin{tabular}{c | cccc}
                                            & \texttt{Weather=sunny} & \texttt{Weather=rain} & \texttt{Weather=cloudy} & \texttt{Weather=snow} \\
                    \hline
                    \texttt{Cavity=true}    & 0.144 & 0.02 & 0.016 & 0.02 \\
                    \texttt{Cavity=false}   & 0.576 & 0.08 & 0.064 & 0.08
                \end{tabular}
            \end{center}
        \end{example}

    \item[Probability density function] \marginnote{Probability density function}
        The probability density function (PDF) of a random variable $X$ is a function $p: \mathbb{R} \rightarrow \mathbb{R}$
        such that:
        \[ \int_{\mathcal{T}_X} p(x) \,dx = 1 \]
        \begin{descriptionlist}
            \item[Uniform distribution] \marginnote{Uniform distribution}
                \[ 
                    p(x) = \text{Unif}[a, b](x) = 
                    \begin{cases}
                        \frac{1}{b-a} & a \leq x \leq b \\
                        0 & \text{otherwise}
                    \end{cases} 
                \]
            \item[Gaussian (normal) distribution] \marginnote{Gaussian (normal) distribution}
                \[ \mathcal{N}(\mu, \sigma^2) = \frac{1}{\sigma\sqrt{2\pi}}e^{\frac{-(x-\mu)^2}{2\sigma^2}} \]

                $\mathcal{N}(0, 1)$ is the standard gaussian.
        \end{descriptionlist}

    \item[Conditional probability] \marginnote{Conditional probability}
        Probability of a prior knowledge with new evidence:
        \[ \prob{a \vert b} = \frac{\prob{a \land b}}{\prob{b}} \]
\end{description}