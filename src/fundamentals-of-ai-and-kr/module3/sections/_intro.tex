\chapter{Introduction}


\section{Uncertainty}
\begin{description}
    \item[Uncertainty] \marginnote{Uncertainty}
        A task is uncertain if it has:
        \begin{itemize}
            \item Partial observations
            \item Noisy or wrong information
            \item Uncertain outcomes of the actions
            \item Complex models
        \end{itemize}

        A purely logic approach leads to:
        \begin{itemize}
            \item Risks falsehood: unreasonable conclusion when applied in practice.
            \item Weak decisions: too many conditions required to make a conclusion.
        \end{itemize}
\end{description}


\subsection{Handling uncertainty}
\begin{descriptionlist}
    \item[Default/non-monotonic logic] \marginnote{Default/non-monotonic logic}
        Works on assumptions.
        An assumption can be contradicted by the evidence.

    \item[Rule-based systems with fudge factors] \marginnote{Rule-based systems with fudge factors}
        Formulated as premise $\rightarrow_\text{prob.}$ effect.
        Have the following issues:
        \begin{itemize}
            \item Locality: how can the probability account all the evidence.
            \item Combination: chaining of unrelated concepts.
        \end{itemize}

    \item[Probability] \marginnote{Probability}
        Assign a probability given the available known evidence.

        Note: fuzzy logic handles the degree of truth and not the uncertainty.
\end{descriptionlist}

\begin{description}
    \item[Decision theory] \marginnote{Decision theory}
        Defined as:
        \[ \text{Decision theory} = \text{Utility theory} + \text{Probability theory} \]
        where the utility theory depends on one's preferences.
\end{description}