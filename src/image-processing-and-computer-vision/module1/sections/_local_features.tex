\chapter{Local features}

\begin{description}
    \item[Correspondence points] \marginnote{Correspondence points}
        Image points projected from the same 3D point from different views of the scene.

        \begin{example}[Homography]
            Align two images of the same scene to create a larger image.
            Homography requires at least 4 correspondences. 
            To find them, it does the following:
            \begin{itemize}
                \item Independently find salient points in the two images.
                \item Compute a local description of the salient points.
                \item Compare descriptions to find matching points.
            \end{itemize}
        \end{example}


    \item[Local invariant features] \marginnote{Local invariant features}
        Find correspondences in three steps:
        \begin{descriptionlist}
            \item[Detection] \marginnote{Detection}
                Find salient points (keypoints).
            
                The detector should have the following properties:
                \begin{descriptionlist}
                    \item[Repeatability] Find the same keypoints across different images.
                    \item[Saliency] Find keypoints surrounded by informative patterns.
                    \item[Fast] As it must scan the entire image. 
                \end{descriptionlist}
            

            \item[Description] \marginnote{Description}
                Compute a descriptor for each salient point based on its neighborhood.
            
                A descriptor should have the following properties:
                \begin{descriptionlist}
                    \item[Invariant] Robust to as many transformations as possible (i.e. illumination, weather, scaling, viewpoint, \dots).
                    \item[Distinctiveness/robustness trade-off] The description should only capture important information around a keypoint and 
                        ignore irrelevant features or noise.
                    \item[Compactness] The description should be concise.
                \end{descriptionlist}


            \item[Matching] \marginnote{Matching}
                Identify the same descriptor across different images.
        \end{descriptionlist}
\end{description}

