\chapter{Spatial filtering}


\section{Noise}

The noise added to a pixel $p$ is defined by $n_k(p)$, 
where $k$ indicates the time step (i.e. noise is different at each time step).
It is assumed that $n_k(p)$ is i.i.d and $n_k(p) \sim \mathcal{N}(0, \sigma)$.

The information of a pixel $p$ is therefore defined as:
\[ I_k(p) = \tilde{I}(p) + n_k(p) \]
where $\tilde{I}(p)$ is the real information.

\begin{description}
    \item[Temporal mean denoising] \marginnote{Temporal mean denoising}
        Averaging $N$ images taken at different time steps.
        \[ 
            \begin{split}
                O(p) &= \frac{1}{N} \sum_{k=1}^{N} I_k(p) \\
                    &= \frac{1}{N} \sum_{k=1}^{N} \Big( \tilde{I}(p) + n_k(p) \Big) \\
                    &= \frac{1}{N} \sum_{k=1}^{N} \tilde{I}(p) + \overbrace{\frac{1}{N} \sum_{k=1}^{N} n_k(p)}^{\mathclap{\text{$\mu = 0$}}} \\
                    &\approx \tilde{I}(p)
            \end{split}
        \]

        \begin{remark}
            As multiple images of the same object are required, this method is only suited for static images.
        \end{remark}

    \item[Spatial mean denoising] \marginnote{Spatial mean denoising}
        Given one image, average across neighboring pixels.

        Let $K_p$ be the pixels in a window around $p$ (included):
        \[ 
            \begin{split}
                O(p) &= \frac{1}{\vert K_p \vert} \sum_{q \in K_p} I(p) \\
                    &= \frac{1}{\vert K_p \vert} \sum_{q \in K_p} \Big( \tilde{I}(q) + n(q) \Big) \\
                    &= \frac{1}{\vert K_p \vert} \sum_{q \in K_p} \tilde{I}(q) + \frac{1}{\vert K_p \vert} \sum_{q \in K_p} n(q) \\
                    &\approx \frac{1}{\vert K_p \vert} \sum_{q \in K_p} \tilde{I}(q)
            \end{split}
        \]

        \begin{remark}
            As the average of neighboring pixels is considered, this method is only suited for uniform regions.
        \end{remark}
\end{description}