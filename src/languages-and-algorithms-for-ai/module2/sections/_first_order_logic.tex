\chapter{First order logic}


\section{Syntax}
\marginnote{Syntax}
The symbols of propositional logic are:
\begin{descriptionlist}
    \item[Constants]
        Known elements of the domain. Do not represent truth values.
        
    \item[Variables]
        Unknown elements of the domain. Do not represent truth values.
    
    \item[Function symbols] 
        Function $f^{(n)}$ applied on $n$ constants to obtain another constant.

    \item[Predicate symbols]
        Function $P^{(n)}$ applied on $n$ constants to obtain a truth value.

    \item[Connectives] $\forall$ $\exists$ $\land$ $\vee$ $\rightarrow$ $\lnot$ $\leftrightarrow$ $\bot$ $($ $)$
\end{descriptionlist}

Using the basic syntax, the following constructs can be defined:
\begin{descriptionlist}
    \item[Term] Denotes elements of the domain.
        \[ t := \texttt{constant} \,|\, \texttt{variable} \,|\, f^{(n)}(t_1, \dots, t_n) \]

    \item[Proposition] Denotes truth values.
        \[
            P := \bot \,|\, P \land P \,|\, P \vee P \,|\,  P \rightarrow P \,|\, P \leftrightarrow P \,|\, 
                \lnot P \,|\, \forall x. P \,|\, \exists x. P \,|\, (P) \,|\, P^{(n)}(t_1, \dots, t_n) 
        \]
\end{descriptionlist}


\begin{description}
    \item[Well-formed formula] \marginnote{Well-formed formula}
        The definition of well-formed formula in first order logic extends the one of
        propositional logic by adding the following conditions:
        \begin{itemize}
            \item If S is well-formed, $\exists X. S$ is well-formed. Where $X$ is a variable.
            \item If S is well-formed, $\forall X. S$ is well-formed. Where $X$ is a variable.
        \end{itemize}
\end{description}