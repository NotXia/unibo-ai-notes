\chapter{Complexity}


\begin{description}
    \item[Complexity class] \marginnote{Complexity class}
        Set of tasks that can be computed within some fixed resource bounds.
\end{description}



\section{Polynomial time}

\begin{description}
    \item[Deterministic time (\DTIME)] \marginnote{Deterministic time (\DTIME)}
        Let $T: \mathbb{N} \rightarrow \mathbb{N}$ and $\mathcal{L}$ be a language.
        $\mathcal{L}$ is in $\DTIME(T(n))$ iff
        there exists a TM that decides $\mathcal{L}$ in time $O(T(n))$.

    \item[Polynomial time (\P)] \marginnote{Polynomial time (\P)}
        The class \P contains all the tasks computable in polynomial time:
        \[ \P = \bigcup_{c \geq 1} \DTIME(n^c) \]

        \begin{remark}
            \P is closed to various operations on programs (e.g. composition of programs)
        \end{remark}

        \begin{remark}
            In practice, the exponent is often small.
        \end{remark}

        \begin{remark}
            \P considers the worst case and is not always realistic.
            Other alternative computational models exist.
        \end{remark}
    
    \item[Church-Turing thesis] \marginnote{Church-Turing thesis}
        Any physically realizable computer can be simulated by a TM with an arbitrary time overhead.

    \item[Strong Church-Turing thesis] \marginnote{Strong Church-Turing thesis}
        Any physically realizable computer can be simulated by a TM with a polynomial time overhead.

        \begin{remark}
            If this thesis holds, the class \P is robust 
            (i.e. does not depend on the computational device)
            and is therefore the smallest class of bounds.
        \end{remark}

    \item[Deterministic time for functions (\FDTIME)] \marginnote{Deterministic time for functions (\FDTIME)}
        Let $T: \mathbb{N} \rightarrow \mathbb{N}$ and 
        $f: \{0, 1\}^* \rightarrow \{0, 1\}^*$.
        $f$ is in $\FDTIME(T(n))$ iff
        there exists a TM that computes it in time $O(T(n))$.

    \item[Polynomial time for functions (\FP)] \marginnote{Polynomial time for functions (\FP)}
        The class \FP is defined as:
        \[ \FP = \bigcup_{c \geq 1} \FDTIME(n^c) \]
        
        \begin{remark}
            It holds that $\forall \mathcal{L} \in \P \Rightarrow f_\mathcal{L} \in \FP$,
            where $f_\mathcal{L}$ is the characteristic function of $\mathcal{L}$.
            Generally, the contrary does not hold.
        \end{remark}
\end{description}



\section{Exponential time}

\begin{description}
    \item[Exponential time (\EXP/\FEXP)] \marginnote{Exponential time (\EXP/\FEXP)}
        The \EXP and \FEXP classes are defined as:
        \[
            \EXP = \bigcup_{c \geq 1} \DTIME\big( 2^{n^c} \big) \hspace{3em} 
            \FEXP = \bigcup_{c \geq 1} \FDTIME\big( 2^{n^c} \big)
        \]

        \begin{theorem}
            The following hold:
            \[ \P \subset \EXP \hspace{3em} \FP \subset \FEXP \]
        \end{theorem}
\end{description}