\chapter{Introduction}

\begin{description}
    \item[Computational task] \marginnote{Computational task}
        Description of a problem.

    \item[Computational process] \marginnote{Computational process}
        Algorithm to solve a task.

        \begin{description}
            \item[Algorithm (informal)] 
                A finite description of 
                elementary and deterministic computation steps. 
        \end{description}
\end{description}


\section{Notations}

\begin{description}
    \item[Set of the first $n$ natural numbers]
        Given $n \in \mathbb{N}$, we have that $[n] = \{ 1, \dots, n \}$.
\end{description}


\subsection{Strings}

\begin{description}
    \item[Alphabet] \marginnote{Alphabet} 
        Finite set of symbols.
    
    \item[String] \marginnote{String} 
        Finite, ordered, and possibly empty tuple of elements of an alphabet.

        The empty string is denoted as $\varepsilon$.
    
    \item[Strings of given length]
        Given an alphabet $S$ and $n \in \mathbb{N}$, we denote with $S^n$ the set of all the strings over $S$ of length $n$.

    \item[Kleene star] \marginnote{Kleene star}
        Given an alphabet $S$, we denote with $S^* = \bigcup_{n=0}^{\infty} S^n$ the set of all the strings over $S$.

    \item[Language] \marginnote{Language}
        Given an alphabet $S$, a language $\mathcal{L}$ is a subset of $S^*$.
\end{description}


\subsection{Tasks encoding}

\begin{description}
    \item[Encoding] \marginnote{Encoding}
        Given a set $A$, any element $x \in A$ can be encoded into a string of the language $\{0, 1\}^*$.
        The encoding of $x$ is denoted as $\enc{x}$ or simply $x$.

    \item[Task function] \marginnote{Task}
        Given two countable sets $A$ and $B$ representing the domain,
        a task can be represented as a function $f: A \rightarrow B$.

        When not stated, $A$ and $B$ are implicitly encoded into $\{0, 1\}^*$.

    \item[Characteristic function] \marginnote{Characteristic function}
        Boolean function of form $f: \{0, 1\}^* \rightarrow \{0, 1\}$.

        Given a characteristic function $f$, the language $\mathcal{L}_f = \{ x \in \{0, 1\}^* \mid f(x) = 1 \}$
        can be defined.

    \item[Decision problem] \marginnote{Decision problem}
        Given a language $\mathcal{M}$, a decision problem is the task of computing a boolean function $f$ 
        able to determine if a string belongs to $\mathcal{M}$ (i.e. $\mathcal{L}_f = \mathcal{M}$).
\end{description}


\subsection{Asymptotic notation}

\begin{description}
    \item[Big O] \marginnote{Big O}
        A function $f: \mathbb{N} \rightarrow \mathbb{N}$ is $O(g)$ if $g$ is an upper bound of $f$.
        \[ f \in O(g) \iff \exists \bar{n} \in \mathbb{N} \text{ such that } \forall n > \bar{n}, \exists c \in \mathbb{R}^+: f(n) \leq c \cdot g(n) \]
    
    \item[Big Omega] \marginnote{Big Omega}
        A function $f: \mathbb{N} \rightarrow \mathbb{N}$ is $\Omega(g)$ if $g$ is a lower bound of $f$.
        \[ f \in \Omega(g) \iff \exists \bar{n} \in \mathbb{N} \text{ such that } \forall n > \bar{n}, \exists c \in \mathbb{R}^+: f(n) \geq c \cdot g(n) \]

    \item[Big Theta]\marginnote{Big Theta}
        A function $f: \mathbb{N} \rightarrow \mathbb{N}$ is $\Theta(g)$ if $g$ is both an upper and lower bound of $f$.
        \[ f \in \Theta(g) \iff f \in O(g) \text{ and } f \in \Omega(g) \]
    
\end{description}