\chapter{Machine learning}

\begin{description}
    \item[Machine learning] \marginnote{Machine learning}
        Application of methods and algorithms to extract patterns from data.
\end{description}

\section{Tasks}
\begin{description}
    \item[Classification] Estimation of a finite number of classes.
    \item[Regression] Estimation of a numeric value.
    \item[Similarity matching] Identify similar individuals.
    \item[Clustering] Grouping individuals based on their similarities.
    \item[Co-occurrence grouping] Identify associations between entities based on the transactions in which they appear together.
    \item[Profiling] Behavior description.
    \item[Link analysis] Analysis of connections (e.g. in a graph).
    \item[Data reduction] Reduce the dimensionality of data with minimal information loss.
    \item[Casual modeling] Understand the connections between events and actions.  
\end{description}


\section{Categories}
\begin{description}
    \item[Supervised learning] \marginnote{Supervised learning}
        Problem where the target(s) is defined.
    \item[Unsupervised learning] \marginnote{Unsupervised learning}
        Problem where no specific target is known.
    \item[Reinforcement learning] \marginnote{Reinforcement learning}
        Learn a policy to generate a sequence of actions.
\end{description}



\section{Data}

\begin{description}
    \item[Dataset] \marginnote{Dataset}
        Set of $N$ individuals, each described by $D$ features. 
\end{description}


\subsection{Data types}

\begin{description}
    \item[Categorical] Values with a discrete domain.
        \begin{description}
            \item[Nominal] \marginnote{Categorical nominal data}
                The values are a set of non-ordered labels.

                \textbf{Operators.} $=$, $\neq$
                \begin{example}
                    Name, surname, zip code.
                \end{example}

            \item[Ordinal] \marginnote{Categorical ordinal data}
                The values are a set of totally ordered labels.

                \textbf{Operators.} $=$, $\neq$, $<$, $>$, $\leq$, $\geq$
                \begin{example}
                    Non-numerical quality evaluations (excellent, good, fair, poor, bad).
                \end{example}
        \end{description}

    \item[Numerical] Values with a continuous domain.
        \begin{description}
            \item[Interval] \marginnote{Numerical interval data}
                Numerical values without an univocal definition of 0 (i.e. 0 is not used as reference).
                It is not reasonable to compare the magnitude of this type of data.

                \textbf{Operators.} $=$, $\neq$, $<$, $>$, $\leq$, $\geq$, $+$, $-$
                \begin{example}
                    Celsius and Fahrenheit temperature scales, CGPA, time, \dots.
                    
                    For instance, there is a $6.25\%$ increase from $16\text{°C}$ to $17\text{°C}$, but
                    converted in Fahrenheit, the increase is of $2.96\%$ (from $60.8\text{°F}$ to $62.6\text{°F}$).
                \end{example}

            \item[Ratio] \marginnote{Numerical ratio data}
                Values with an absolute 0 point.

                \textbf{Operators.} $=$, $\neq$, $<$, $>$, $\leq$, $\geq$, $+$, $-$
                \begin{example}
                    Kelvin temperature scale, age, income, length.

                    For instance, there is a $10\%$ increase from 100\$ to 110\$.
                    Converted in euro (1\geneuro = 1.06\$), the increase is still of $10\%$ (from $94.34\geneuro$ to $103.77\geneuro$).
                \end{example}
        \end{description}
\end{description}


\subsection{Transformations}
\begin{center}
    
    \begin{tabular}{c|c|>{\raggedright\arraybackslash}m{8cm}}
        \hline
        \multicolumn{2}{c|}{\textbf{Data type}} & \textbf{Transformation} \\
        \hline
        \multirow{2}{*}{Categorical}    & Nominal  & One-to-one transformations \\
        \cline{2-3}
                                        & Ordinal  & Order preserving transformations (i.e. monotonic functions) \\
        \hline
        \multirow{2}{*}{Numerical}      & Interval & Linear transformations \\
        \cline{2-3}
                                        & Ratio    & Any mathematical function, standardization, variation in percentage \\
        \hline
    \end{tabular}
\end{center}


% \subsection{Dataset characteristics}
% \begin{description}
%     \item[Dimensionality] 
%     \item[Sparsity] 
%     \item[Missing data] 
%     \item[Resolution] 
% \end{description}


\subsection{Dataset format}
\begin{description}
    \item[Relational table] \marginnote{Relational table}
        The attributes of each record are the same.
    
    \item[Data matrix] \marginnote{Data matrix}
        Matrix with $N$ rows (entries) and $D$ columns (attributes).
    
    \item[Sparse matrix] \marginnote{Sparse matrix}
        Data matrix with lots of zeros.
        \begin{example}[Bag-of-words]
            Each row represents a document, each column represents a term.
            The $i,j$-th cell contains the frequency of the $j$-th term in the $i$-th document.
        \end{example}
    
    \item[Transactional data] \marginnote{Transactional data}
        Each record contains a set of objects (not necessarily a relational table).
    
    \item[Graph data] \marginnote{Graph data}
        Set of nodes and edges.
    
    \item[Ordered data] \marginnote{Ordered data}
        e.g. temporal data.
\end{description}


\subsection{Data quality}
\begin{description}
    \item[Noise] \marginnote{Noise}
        Alteration of the original values.

    \item[Outliers] \marginnote{Outliers}
        Data that considerably differ from the majority of the dataset.
        May be caused by noise or rare events.

        Box plots can be used to visually detect outliers.

    \item[Missing values] \marginnote{Missing values}
        Data that have not been collected.
        Sometimes they are not easily recognizable 
        (e.g. when special values are used to mark missing data instead of \texttt{null}).

        Can be handled in different ways:
        \begin{itemize}
            \item Ignore the records with missing values.
            \item Estimate or default missing values.
            \item Ignore the fact that some values are missing (not always applicable).
            \item Insert all the possible values and weigh them by their probability.
        \end{itemize}

    \item[Duplicated data] \marginnote{Duplicated data}
        Data that may be merged.
\end{description}