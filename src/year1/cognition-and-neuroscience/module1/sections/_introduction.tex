\chapter{Introduction}



\section{Definitions}

\begin{description}
    \item[Neuroscience] \marginnote{Neuroscience}
        Study of the nervous system (structure aspects) on various levels of detail:
        \begin{descriptionlist}
            \item[Molecular] Proteins and molecular signaling of the nervous system.
            \item[Cellular] Morphological and physiological properties of neurons.
            \item[Neural system] Creation and functioning of networks of neurons.
        \end{descriptionlist}


    \item[Cognition] \marginnote{Cognition}
        Mental processes (function aspects) that react to inputs. 
        It involves processes regarding the acquisition, storage, manipulation, and retrieval of information.
        \begin{descriptionlist}
            \item[Perception] Information from the environment.
            \item[Attention] Focus on a specific stimulus in the environment.
            \item[Learning] Merging new information with prior knowledge.
            \item[Memory] Encoding, storing, and retrieving information.
            \item[Action] Interact with the environment using perceived information.
            \item[Language] Understanding and producing spoken or written thoughts.
            \item[Higher reasoning] Decision-making and problem-solving.
        \end{descriptionlist}


    \item[Biomimicry] \marginnote{Biomimicry}
        Solving problems by taking inspiration from elements of nature.

        As proof of general intelligence\footnote{\includegraphics[width=1cm]{img/doubt.png}}, 
        the human brain is taken as the model for artificial intelligence.
        Moreover, a successful brain-inspired AI application can 
        provide a possibly plausible explanation of the functioning of the brain.
        
        However, a brain differs from a computer in many aspects:
        \begin{itemize}
            \item Hardware and software are distinct while mind and brain are not.
            \item Machines learn by exploiting the capability of using a large memory
                while brains have limited capacity but high generalization ability.
            \item Brains produce both electrical and biochemical signals and 
                have feedforward, feedback, and recurrent connections
                while machines typically only employ feedforward connections.
        \end{itemize}

        \begin{description}
            \item[Structure emulation] 
                Mimic or reverse engineer the structure of the brain (e.g. Blue Brain Project).

            \item[Function emulation] 
                Mimic a neural system on the algorithmic level (e.g. Deep Mind).
        \end{description}


    \item[Cognitive neuroscience] \marginnote{Cognitive neuroscience}
        Study of the relationship between the physical brain and the intangible mind (thoughts, ideas).
        In other words, it studies the relationship between structure and function.

        \begin{casestudy}[Severed Corpus Callosum \href{https://www.youtube.com/watch?v=lfGwsAdS9Dc}{\texttt{video}}]
            Normally, the right and left hemispheres of the brain can communicate. 
            Moreover, the left visual field is sent to the right hemisphere and 
            the right visual field is sent to the left hemisphere.

            In patients where the hemispheres are split, a text shown on the right visual side is recognized as 
            the speech capabilities are located in the left hemisphere,
            while a text shown on the left visual side does not trigger any speech reaction.
        \end{casestudy}
\end{description}



\section{Neuroscience history}

Two main schools of thought emerged and are still the subject of ongoing debates:
\begin{descriptionlist}
    \item[Localizationism] \marginnote{Localizationism}
        Specific regions of the brain are responsible for particular faculties.

        Assuming localizationism, 52 distinct regions with different neurons can be identified.

    \item[Aggregate field theory] \marginnote{Aggregate field theory}
        The brain works as a whole for mental functions.
\end{descriptionlist}


\subsection{Neuron doctrine}
\marginnote{Neuron doctrine}
The nervous system is made of a discrete amount of individual neurons (and not a continuous tissue).

\begin{description}
    \item[Principle of dynamic polarization] 
        Electrical signals in a neuron flow only in a single direction.

    \item[Principle of connectional specificity] 
        Neurons do not connect randomly but make specific connections at particular contact points.

    \item[Synapse] \marginnote{Synapse}
        Point of contact of two neurons. A synapse can be chemical or electrical.
\end{description}



\section{Cognitive science history}

\begin{description}
    \item[Rationalism] \marginnote{Rationalism}
        All knowledge can be derived through reasoning, without sensory experiences.

    \item[Empiricism] \marginnote{Empiricism}
        The brain starts as a blank slate and knowledge is added through sensory experiences.

    \item[Associationism] \marginnote{Associationism}
        Inspired by empiricism.
        Learning happens by correlating individual experiences (e.g. actions followed by a reward will be repeated).

    \item[Behaviorism] \marginnote{Behaviorism}
        Inspired by empiricism. 
        Everyone has the same neural basis that is improved through learning.
        Learning only involves observable behaviors.
\end{description}

\begin{remark}
    Associationism and behaviorism are not able to explain all types of learning (e.g. language).
\end{remark}

\begin{description}
    \item[Cognitivism] \marginnote{Cognitivism}
        The psychological and biological levels of an individual cannot be separated.
        Learning is based on the biology of the neurons.
\end{description}