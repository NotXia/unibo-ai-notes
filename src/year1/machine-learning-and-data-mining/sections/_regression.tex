\chapter{Regression}

\begin{description}
    \item[Linear regression] \marginnote{Linear regression}
        Given:
        \begin{itemize}
            \item A dataset $\matr{X}$ of $N$ rows and $D$ features.
            \item A response vector $\vec{y}$ of $N$ continuous values.
        \end{itemize}
        We want to learn the parameters $\vec{w} \in \mathbb{R}^D$ such that:
        \[ \vec{y} \approx \matr{X}\vec{w}^T \]

    \item[Mean squared error] \marginnote{Mean squared error}
        To find the parameters for linear regression,
        we minimize as loss function the mean squared error:
        \[  
            \mathcal{L}(\vec{w}) = \Vert \matr{X}\vec{w}^T - \vec{y} \Vert^2    
        \]
        Its gradient is:
        \[ \nabla\mathcal{L}(\vec{w}) = 2\matr{X}^T(\matr{X}\vec{w}^T - \vec{y}) \]
        Constraining it to 0, we obtain the problem:
        \[ \matr{X}^T\matr{X}\vec{w}^T = \matr{X}^T\vec{y} \]
        If $\matr{X}^T\matr{X}$ is invertible, this can be solved analytically but could lead to overfitting.
        Numerical methods are therefore more suited.

        Note that:
        \begin{itemize}
            \item MSE is influenced by the magnitude of the data.
            \item It measures the fitness of a model in absolute terms.
            % \item It is suited to compare different models.
        \end{itemize}

    \item[Coefficient of determination] \marginnote{Coefficient of determination}
        Given:
        \begin{itemize}
            \item The mean of the observed data: $y_\text{avg} = \frac{1}{N} \sum_i \vec{y}_i$.
            \item The sum of the squared residuals: $SS_\text{res} = \sum_i (\vec{y}_i - \vec{w}^T\vec{x}_i)^2$.
            \item The total sum of squares: $SS_\text{tot} = \sum_i (\vec{y}_i - y_\text{avg})^2$.
        \end{itemize}
        The coefficient of determination is given by:
        \[ \text{R}^2 = 1 - \frac{SS_\text{res}}{SS_\text{tot}} \]

        Intuitively, $\text{R}^2$ compares the model with a horizontal straight line ($y_\text{avg}$).
        When $\text{R}^2 = 1$, the model has a perfect fit.
        When $\text{R}^2$ is outside the range $[0, 1]$, then the model is worse than a straight line.

        Note that:
        \begin{itemize}
            \item $\text{R}^2$ is a standardized index.
            \item $\text{R}^2$ tells how well the variables of the predictor can explain the variation in the target.
            \item $\text{R}^2$ is not suited for non-linear models.
        \end{itemize}

    \item[Polynomial regression] \marginnote{Polynomial regression}
        Find a polynomial instead of a hyperplane.
\end{description}