\chapter{Introduction}

\begin{description}
    \item[Wall-clock time] \marginnote{Wall-clock time}
        Time taken to run a program from start to finish.

    \item[High performance computing (HPC)] \marginnote{High performance computing (HPC)}
        Specialized hardware aiming to reduce wall-clock time (e.g., super-computer). A program is split into strongly coupled sub-problems.

    \item[High throughput computing (HTC)] \marginnote{High throughput computing (HTC)} 
        Commodity hardware that guarantees a high job throughput (e.g., cloud computing). A program is split into loosely coupled sub-problems that are not necessarily related to each other.
\end{description}

\begin{remark}
    A system with two units of the same processor with the clock halved is usually more power efficient than a system with a single unit at full speed.
\end{remark}

\begin{description}
    \item[Parallel programming steps] 
        The typical steps to write a parallel program is the following:
        \begin{enumerate}
            \item Decompose the main problem into sub-problems.
            \item Distribute the sub-problems to the execution units.
            \item Solve each sub-problem.
            \item Merge the sub-solutions.
        \end{enumerate}

        \begin{description}
            \item[Embarrassingly parallel problem] \marginnote{Embarrassingly parallel problem}
                Problem that can be split in completely independent sub-problems.

            \begin{remark}
                Typically, parallelizing a sequential algorithm is not straightforward.
            \end{remark}
        \end{description}
\end{description}