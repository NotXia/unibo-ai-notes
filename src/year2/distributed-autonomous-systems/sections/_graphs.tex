\chapter{Graphs}


\section{Definitions}

\begin{description}
    \item[Directed graph (digraph)] \marginnote{Directed graph}
        Pair $G = (I, E)$ where $I=\{1, \dots, N\}$ is the set of nodes and $E \subseteq I \times I$ is the set of edges.

    \item[Undirected graph] \marginnote{Undirected graph}
        Digraph where $\forall i,j: (i, j) \in E \Rightarrow (j, i) \in E$.

    \item[Subgraph] \marginnote{Subgraph}
        Given a graph $(I, E)$, $(I', E')$ is a subgraph of it if $I' \subseteq I$ and $E' \subset E$.
        \begin{description}
            \item[Spanning subgraph] Subgraph where $I' = I$.
        \end{description}

    \item[In-neighbor] \marginnote{In-neighbor}
        A node $j \in I$ is an in-neighbor of $i \in I$ if $(j, i) \in E$.

        \begin{description}
            \item[Set of in-neighbors] \marginnote{Set of in-neighbors}
                The set of in-neighbors of $i \in I$ is the set:
                \[ \mathcal{N}_i^\text{IN} = \{ j \in I \mid (j, i) \in E \} \] 

            \item[In-degree] \marginnote{In-degree}
                Number of in-neighbors of a node $i \in I$:
                \[ \indeg[i] = | \mathcal{N}_i^\text{IN} | \] 
        \end{description}

    \item[Out-neighbor] \marginnote{Out-neighbor}
        A node $j \in I$ is an out-neighbor of $i \in I$ if $(i, j) \in E$.

        \begin{description}
            \item[Set of out-neighbors] \marginnote{Set of in-neighbors}
                The set of out-neighbors of $i \in I$ is the set:
                \[ \mathcal{N}_i^\text{OUT} = \{ j \in I \mid (i, j) \in E \} \] 

            \item[Out-degree] \marginnote{Out-degree}
                Number of out-neighbors of a node $i \in I$:
                \[ \outdeg[i] = | \mathcal{N}_i^\text{OUT} | \] 
        \end{description}


    \item[Balanced digraph] \marginnote{Balanced digraph}
        A digraph is balanced if $\forall i \in I: \indeg[i] = \outdeg[i]$.

    \item[Periodic graph] \marginnote{Periodic graph}
        Graph where there exists a period $k > 1$ that divides the length of any cycle.

        \begin{remark}
            A graph with self-loops is aperiodic.
        \end{remark}

    \item[Strongly connected digraph] \marginnote{Strongly connected digraph}
        Digraph where each node is reachable from any node.

    \item[Connected undirected graph] \marginnote{Connected undirected graph}
        Undirected graph where each node is reachable from any node.    

    \item[Weakly connected digraph] \marginnote{Weakly connected digraph}
        Digraph where its undirected version is connected.
\end{description}



\section{Weighted digraphs}

\begin{description}
    \item[Weighted digraph] \marginnote{Weighted digraph}
        Triplet $G=(I, E, \{a_{i, j}\}_{(i,j) \in E})$ where $(I, E)$ is a digraph and $a_{i,j} > 0$ is a weight for the edge $(i,j)$.

        \begin{description}
            \item[Weighted in-degree] \marginnote{Weighted in-degree}
                Sum of the weights of the inward edges:
                \[ \indeg[i] = \sum_{j=1}^N a_{j, i} \]
            \item[Weighted out-degree] \marginnote{Weighted out-degree}
                Sum of the weights of the outward edges:
                \[ \outdeg[i] = \sum_{j=1}^N a_{i, j} \]
        \end{description}


    \item[Weighted adjacency matrix] \marginnote{Weighted adjacency matrix}
        Non-negative matrix $\matr{A}$ such that $\matr{A}_{i,j} = a_{i,j}$:
        \[
            \begin{cases}
                \matr{A}_{i,j} > 0 & \text{if $(i, j) \in E$} \\
                \matr{A}_{i, j} = 0 & \text{otherwise}
            \end{cases}
        \]

    \item[In/out-degree matrix] \marginnote{In/out-degree matrix}
        Matrix where the diagonal contains the in/out-degrees:
        \[
            \matr{D}^\text{IN} = \begin{bmatrix}
                \indeg[1] & 0 & \cdots & 0 \\
                0 & \indeg[2] \\
                \vdots & & \ddots \\
                0 & \cdots & 0 & \indeg[N] \\
            \end{bmatrix}
            \qquad
            \matr{D}^\text{OUT} = \begin{bmatrix}
                \outdeg[1] & 0 & \cdots & 0 \\
                0 & \outdeg[2] \\
                \vdots & & \ddots \\
                0 & \cdots & 0 & \outdeg[N] \\
            \end{bmatrix}
        \]

        \begin{remark}
            Given a digraph with adjacency matrix $\matr{A}$, its reverse digraph has adjacency matrix $\matr{A}^T$.
        \end{remark}

        \begin{remark}
            It holds that:
            \[ 
                \matr{D}^\text{IN} = \text{diag}(\matr{A}^T \vec{1}) 
                \quad
                \matr{D}^\text{OUT} = \text{diag}(\matr{A} \vec{1})
            \]
            where $\vec{1}$ is a vector of ones.
        \end{remark}

        \begin{remark}
            A digraph is balanced iff $\matr{A}^T \vec{1} = \matr{A} \vec{1}$.
        \end{remark}
\end{description}



\section{Laplacian matrix}

\begin{description}
    \item[(Out-degree) Laplacian matrix] \marginnote{Laplacian matrix}
        Matrix $\matr{L}$ defined as:
        \[ \matr{L} = \matr{D}^\text{OUT} - \matr{A} \]

        \begin{remark}
            The vector $\vec{1}$ is always an eigenvector of $\matr{L}$ with eigenvalue $0$:
            \[ \matr{L}\vec{1} = (\matr{D}^\text{OUT} - \matr{A})\vec{1} = \matr{D}^\text{OUT}\vec{1} - \matr{D}^\text{OUT}\vec{1} = 0 \]
        \end{remark}

    \item[In-degree Laplacian matrix] \marginnote{In-degree Laplacian matrix}
        Matrix $\matr{L}^\text{IN}$ defined as:
        \[ \matr{L}^\text{IN} = \matr{D}^\text{IN} - \matr{A}^T \]

        \begin{remark}
            $\matr{L}^\text{IN}$ is the out-degree Laplacian of the reverse graph.    
        \end{remark}
\end{description}