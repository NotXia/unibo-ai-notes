\chapter{Human rights}


\begin{description}
    \item[Human rights] \marginnote{Human rights}
        Human rights, primarily intended as ethical demands, are related to individual freedoms. They can be negative liberties (i.e., require non-interference from third-parties) and positive liberties (i.e., require active provisioning).

        \begin{remark}
            Not all human rights are necessarily laws as they might not always be legally enforceable. 
        \end{remark}
\end{description}


\section{List of rights}

\begin{description}
    \item[Freedom and dignity] 
        AI systems could undermine this right if used for surveillance, profiling, automated assessment, manipulation, or interference.

    \item[Right to equality and non-discrimination]
        Digitally disadvantaged individuals might be excluded from accessing AI system or exploited. AI systems themselves can be biased and reproduce existing discriminatory practices.

    \item[Right to privacy]
        Related to the right of a person to make autonomous decisions, and to have control of the data collected and how it is processed. 

    \item[Right to life, liberty, and security]
        Protection of the physical and digital (i.e., digital memories) integrity.

    \item[Right to property]
        In the context of information systems, it includes the right of portability of the data from a platform to another.

    \item[Freedom of assembly and association]

    \item[Right to an effective remedy]
        AI systems can support judicial proceedings. However, its application should not be too mechanical and there should be the possibility for automated decisions to be reviewed by humans.

    \item[Right to hearing]
    
    \item[Presumption of innocence]
    
    \item[Freedom of opinion, expression, and information]
        AI systems should not undermine the freedom of expression. It could also be used to moderate online interactions and filter out hate speech or fake news.

    \item[Right to take part in government]
        AI system should not be used to undermine political rights through surveillance, pervasive data collection, opinion polarization, \dots

    \item[Right to social security]
    
    \item[Right to work]
        Workers that are substituted with AI systems should be protected.

    \item[Right to adequate living standards]
        Deployment of AI systems should be more concerned with regard to environmental impacts.

    \item[Right to education]
        AI system in education should not deprive students of human relationships.

    \item[Right to culture]
        Content generated by AI can harm content creators and reduce originality in the overall cultural landscape.
\end{description}