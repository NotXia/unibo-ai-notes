\chapter{Introduction to ethics}



\section{Morality}

\begin{description}
    \item[Morality] \marginnote{Morality}
        There is no widely agreed definition of morality. On a high-level, it refers to norms to determine which actions are right or wrong.  
\end{description}


\subsection{Conventional and critical morality}

\begin{description}
    \item[Positive (conventional) morality] \marginnote{Positive morality}
        Rules and principles created by humans that are widely accepted in a culture/society.

        \begin{remark}
            Conventional morality can change between different societies.
        \end{remark}
        
        \begin{remark}
            In principle, the starting moral values of an individual are those of the society it was born into.
        \end{remark}

    \item[Critical morality] \marginnote{Critical morality}
        Moral standards that are independent of conventional morality and are correct in general. It does not come from social agreements or beliefs.

        \begin{remark}
            Popular moral views are not necessarily true. Critical morality can be used as a ground-truth to determine whether conventional morality is correct.
        \end{remark}
\end{description}


\subsection{Branches of moral philosophy}

\begin{description}
    \item[Value theory] \marginnote{Value theory}
        Field that studies values (i.e., source of goodness and badness). Some research questions are: ``what is the good life?'', ``what is happiness?'', \dots

    \item[Normative ethics] \marginnote{Normative ethics}
        Field that studies what is morally required and how one should act. Some research questions are: ``what makes right actions right?'', ``do the ends justify the means?'', \dots

    \item[Non-normative ethics] \phantom{}
        \begin{description}
            \item[Descriptive ethics] \marginnote{Descriptive ethics}
                Field that uses scientific techniques to study how people reason and act.

            \item[Meta-ethics] \marginnote{Meta-ethics}
                Field that analysis the language, concepts, and methods of reasoning in normative ethics. Some research questions are ``can ethical judgments be true or false?'', ``does morality correspond to facts in the world?'', \dots
        \end{description}
\end{description}


\subsection{Morality vs other normative systems} \label{sec:morality_other_systems}

\begin{description}
    \item[Laws] \marginnote{Morality vs laws} 
        Laws are not necessarily in line with what should be morally correct. Some legal actions are immoral (e.g., cheating) and some illegal actions are moral (e.g., criticizing a dictator).

    \item[Etiquette] \marginnote{Morality vs etiquette}
        Standards of etiquette and good manners are not necessarily moral (e.g., forks should be put to the left of the plate, but it is not immoral to put them to the right).

    \item[Self-interest] \marginnote{Morality vs self-interest} 
        Sometimes, morality requires to sacrifice our well-being and, vice versa, immoral acts can improve our life.

    \item[Tradition] \marginnote{Morality vs tradition} 
        Practices that have been consolidated through time are not necessarily moral.
    
    \item[Religion] \marginnote{Morality vs religion} 
        Religions are based on the fact that there is a higher authority that created the set of moral norms. However, it is not clear whether God commands something because it considers them moral or things are moral because God commanded them.
\end{description}

% \begin{description}
%     \item[Ethics] 
%         Determine what is morally required.

%     \item[Meta-ethics] 
%         Study nature, scope, and meaning of moral judgement.
% \end{description}


\subsection{Absolutism and relativism}

\begin{description}
    \item[Absolutism] \marginnote{Absolutism}
        There is a single true ethics.

    \item[Relativism] \marginnote{Relativism}
        Judgment is relative to particular frameworks and attitudes.
\end{description}



\section{Consequentialism}

\begin{description}
    \item[Optimific action] \marginnote{Optimific action}
        Action that produces the best overall results.

    \item[Consequentialism] \marginnote{Consequentialism}
        Family of theories that consider an action morally required if and only if it is optimific.

        \begin{remark}
            Consequentialism sees morality as an optimization problem.
        \end{remark}
\end{description}


\subsection{Act utilitarianism}

\begin{description}
    \item[Act utilitarianism] \marginnote{Act utilitarianism}
        Instance of consequentialism that considers well-being the only thing that is intrinsically valuable.

    \item[Principle of utility] \marginnote{Principle of utility}
        Moral standard of act utilitarianism that considers an action morally required if and only if it is, among all the possibilities, the one that improves the overall world's well-being the most.
\end{description}

\begin{remark}
    Strengths of act utilitarianism are:
    \begin{itemize}
        \item It is egalitarian and impartial as everyone's utility counts the same. Every person and every animal is part of the moral community.
        \item It justifies some basic moral intuitions (e.g., slavery is, in general, not optimific).
        \item It has by design a way of dealing with moral conflicts (i.e., choose the action that maximizes well-being).
        \item It is flexible as actions violating some moral rules can be performed if they increase the overall well-being.
    \end{itemize}
\end{remark}

\begin{remark}
    Problems of act utilitarianism are:
    \begin{itemize}
        \item It is too demanding on the individuals as it requires constant self-sacrifice.
        \item It does not provide a decision procedure or a way to assess decisions.
        \item It has no room for partiality (i.e., a family member is as important as a stranger).
        \item If the majority of society is against a minority group, unjust actions against the minority increases the overall world's well-being.
    \end{itemize}
\end{remark}


\subsection{Rule utilitarianism}

\begin{description}
    \item[Optimific social rule] \marginnote{Optimific social rule}
        Rule based on the idea that in the hypothetical case (nearly) everyone in a society were to accept it, the results would be optimific. 

    \item[Rule utilitarianism] \marginnote{Rule utilitarianism}
        An action is morally right if it is required by an optimific social rule.

\end{description}

\begin{remark}
    Rule utilitarianism allows some degree of partiality. Moreover, an optimific action itself is not necessarily morally required if it is against an optimific social rule (e.g., it might be an optimific action to torture a prisoner, however, torture is forbidden by optimific social rules as it is more beneficial in the long term).
\end{remark}



\section{Deontology}

\begin{description}
    \item[Deontology] \marginnote{Deontology}
        Ethical theory which states that actions are good or bad independently of their consequences. An action is morally right if it conforms with a moral norm.
\end{description}


\subsection{Kantian ethics}

\begin{description}
    \item[Kantian ethics] 
        Ethical theory based on fairness and consistency.

        \begin{remark}
            Two popular morality tests that however fail to assess fairness and consistency are:
            \begin{itemize}
                \item ``What if everyone did that?'', which can be interpreted as ``if disastrous results would occur if everyone did X, then X is immoral''. This test is sensitive to how the action is described as the same action can be morally right and wrong depending on its description.
                \item ``How would you like it if I did that to you?'', which is the golden rule. This test makes morality depend on a person's desire, which fails when the principles of a person are wrong (e.g., some nazis would accept to be killed if they discovered Jewish ancestors).
            \end{itemize}
        \end{remark}

    \item[Maxim] \marginnote{Maxim}
        Subjective principle that one gives to itself when performing an action. In other words, it states what one is about to do and why.

        \begin{remark}
            Maxims are related to the individual's intentions.
        \end{remark}

        \begin{remark}
            Differently from consequentialism, morality in Kantian ethics does not depend on the results but on the reason of the actions. Two people doing the same action with identical results might be guided by different maxims.
        \end{remark}

        \begin{remark}
            By focusing on maxims, morality only depends on what is within our control. Result-oriented approaches are instead not always predictable and therefore should not be used.
        \end{remark}

    \item[Universalizable maxim] \marginnote{Universalizable maxim}
        A maxim is universalizable if it passes the following test:
        \begin{enumerate}
            \item Formulate my maxim clearly.
            \item Hypothesize a world where everyone supports and acts according to that maxim.
            \item Ask whether the goal of my action can be achieved in such a world.
        \end{enumerate}

        \begin{remark}
            Differently from consequentialism, the question asked aims at determining if our goal can be reached instead of determining whether the world would be better.
        \end{remark}

        \begin{remark}
            If a maxim is universalizable, it would mean that everyone could support it and our actions are not making an unfair exception for ourselves.
        \end{remark}

    \item[Principle of universalizability] \marginnote{Principle of universalizability}
        Basis of Kantian ethics. An action is acceptable if and only if its maxim is universalizable.

        Acting on non-universalized maxims makes us inconsistent. Immoral actions can be therefore considered irrational.
\end{description}

\begin{remark}[Amoralist's challenge]
    Amoralists are those that believe in right and wrong but act disregarding morality nevertheless. They challenge the fact that immoral actions are irrational as follows:
        \begin{enumerate}
            \item People have a reason to do something if it will give them what they want.
            \item Moral duty sometimes fails to give people what they want.
            \item Therefore, people sometimes do not have a reason to act morally.
            \item If there is no reason to act morally, violating moral duties is rational.
            \item Therefore, it is rational to violate moral duties.
        \end{enumerate}
\end{remark}

\begin{description}
    \item[Hypothetical imperatives] \marginnote{Hypothetical imperatives}
        Imperatives that require us to do what is needed to reach our goal. They are dependent on the individual's needs and disregarding them makes one irrational.

        \begin{remark}
            Hypothetical imperatives change if one's desires change.
        \end{remark}

        \begin{remark}
            Hypothetical imperatives are the way Kantian ethics deals with the amoralist's challenge.
        \end{remark}

    \item[Categorical imperatives] \marginnote{Categorical imperatives}
        Imperatives that do not depend on a single individual but are applicable to every rational being. Categorical imperatives command to do things that one might want or not want to do. Disregarding them makes one irrational.

        \begin{remark}
            According to Kant's \textit{argument for the irrationality of immorality}, moral duties are categorical imperatives:
            \begin{enumerate}
                \item If you are rational, then you are consistent.
                \item If you are consistent, then you obey the principle of universalizability.
                \item If you obey the principle of universalizability, then you act morally.
                \item Therefore, if you are rational, you act morally.
                \item Therefore, if you act immorally, you are irrational.
            \end{enumerate}
        \end{remark}

        \begin{description}
            \item[Principle of humanity] \marginnote{Principle of humanity}
                Alternative interpretation of categorical imperatives. It states that one should treat humanity as an end and never only as means. In other words, one should never treat people without considering their dignity (i.e., intended as one's reason and autonomy).
        \end{description}
\end{description}

\begin{remark}
    Kantian ethics revolves around integrity, which requires living in harmony with the principle one believes in. However, it does not capture the fact that if these principles are flawed, it would be morally more correct to have less integrity.
\end{remark}


\subsection{David Ross's prima facie duties}

\begin{description}
    \item[Prima facie duties] \marginnote{Prima facie duties}
        Ethics theory which states that everyone has obligations that are however defeasible in case of tensions and conflicts. These obligations are:
        \begin{descriptionlist}
            \item[Fidelity] Keep promises, be honest and truthful.
            \item[Reparation] Make amends when we have wronged someone else.
            \item[Gratitude] Be grateful to others whose actions benefit us. Try to return the favor.
            \item[Non-maleficence] Refrain from harming others.
            \item[Beneficence] Be kind and improve others.
            \item[Self-improvement] Improve our own health, wisdom, security, happiness, \dots
            \item[Justice] Be fair and distribute benefits and burdens equably.
        \end{descriptionlist}
\end{description}


\subsection{Nietzsche's critique of ethics}

\begin{description}
    \item[Superior human] \marginnote{Superior human}
        Who is beyond the morality of the common people. It only has duties toward equals and can treat beings of lower rank as it wishes.
\end{description}



\section{Proceduralism}

\begin{description}
    \item[Proceduralism] \marginnote{Proceduralism}
        Approach to ethics that does not start by making assumptions on any basic moral views but rather follows a procedure to show that they are morally right.

        \begin{remark}
            The golden rule, rule consequentialism, Kant's principle of universalizability are all instances of proceduralism.
        \end{remark}

        \begin{remark}
            These theories are still originated from Kantian ethics.
        \end{remark}
\end{description}


\subsection{Contractarianism}

\begin{description}
    \item[Contractarianism (political)] \marginnote{Contractarianism (political)}
        Political theory which states that laws are just if and only if they would be accepted by free, equal, and rational people.

        \begin{remark}
            Rationality implies that everyone will cooperate limiting self-interest. In this way, everyone will give up a luxurious life but also avoid a terrible one.
        \end{remark}

    \item[Contractarianism (moral)] \marginnote{Contractarianism (moral)}
        Ethical theory which states that actions are morally right if and only if they would be accepted by free, equal, and rational people, on the condition that everyone obeys to these rules.
\end{description}

\begin{description}
    \item[Prisoner's dilemma] \marginnote{Prisoner's dilemma}
        Situation where the best outcome would be obtained if everyone stops pursuing its self-interest.

        \begin{table}[H]
            \caption{
                \parbox[t]{0.7\linewidth}{
                    Scenario that the dilemma takes inspiration from: two prisoners are interrogated separately; they can either stay silent (cooperate) or snitch the other (betray). The numbers are the years in prison each of them would get.
                }
            }
            \centering
            \begin{tabular}{l|cc}
                \toprule
                            & Cooperate & Betray    \\
                \midrule
                Cooperate   & 2, 2      & 6, 0      \\
                Betray      & 0, 6      & 4, 4      \\
                \bottomrule
            \end{tabular}
        \end{table}

    \item[State of nature] \marginnote{State of nature}
        Situation where there is no government, central authority, or any group than enforces its will on others. In such a situation, everyone acts to maximize its own self-interest. However, the effect is that everyone will be in worse conditions.

        To escape from the state of nature, two things are needed:
        \begin{itemize}
            \item Beneficial rules that require cooperation and punish betrayal.
            \item An enforcer that ensures the rules are obeyed.
        \end{itemize}
\end{description}


\begin{description}
    \item[Contractarianism characteristics] \phantom{} 
        \begin{itemize}
            \item Morality is a social phenomenon: moral rules are basically rules of cooperation. There are no self-regarding moral duties, so any action that does not have bearing on others is morally right.
            \item Basic moral rules are justified.
                \begin{descriptionlist}
                    \item[Veil of ignorance] \marginnote{Veil of ignorance}
                        A test where rational people choose social rules solely based on their basic human needs (without other factors such as religion, ethnicity, sex, \dots). In this scenario, it is expected that choices are made based on two principles:
                        \begin{enumerate}
                            \item Each person will prioritize basic liberties, which will match those of everyone.
                            \item Social and economic inequalities are allowed if everyone has equal access to those positions and the benefits should be aimed to the least advantaged members of society. 
                        \end{enumerate}
                        Overall, what will be selected is going to match the basic moral rules.
                \end{descriptionlist}
            \item There is a procedure to determine if an action is right or wrong: ask whether free, equal, and rational people would agree to rules that allow that action.
            \item Contractarianism justifies the origin of morality as originated from the same society we live in, but in a more rational and free version.
            \item Moral rules can be violated when people stop cooperating (i.e., when there is a state of nature).
            \item Contractarianism justifies the basic moral duty to obey the law, as otherwise there would be a state of nature. For the same reason, it justifies legal punishment and gives the state the authority in criminal law.
            \item By definition, contractarianism justifies breaking the law through non-violent civil disobedience when the law itself fails to set fair cooperation conditions.
        \end{itemize}
\end{description}


\subsection{Habermas' discourse ethics}

\begin{description}
    \item[Discourse ethics] \marginnote{Discourse ethics}
        Ethical theory which states that an action is justified if and only if all those affected could accept it in a reasonable discourse (i.e., everyone involved is considered equal and free).

        \begin{remark}
            It is assumed that people are able to engage in a discourse and converge to a common choice.
        \end{remark}
\end{description}



\section{Virtue ethics}

\begin{description}
    \item[Virtue ethics] \marginnote{Virtue ethics}
        Family of theories that considers an action morally right if and only if a virtuous person (i.e., an ideal character, a role model) would do.

        \begin{remark}
            Virtue ethics rejects the idea of having a simple test to determine what is morally right.
        \end{remark}

        \begin{remark}
            A virtuous person is different from one that habitually do the right thing. The former compared to the latter is motivated in doing so and has a greater sensitivity.
        \end{remark}

    \item[Moral wisdom] \marginnote{Moral wisdom}
        Know-how that tells us what is morally right. It is developed through experience starting from the moral rules one have been taught as a child.

        \begin{remark}[Emotions in moral understanding]
            Emotions can provide a signal to determine the rightness of actions:
            \begin{itemize}
                \item Emotions suggest what is morally relevant in a given situation (e.g., compassion, sympathy, kindness, \textit{etc.} are virtues a moral person should have).
                \item Emotions tell us what is morally right and wrong (e.g., anxiety when doing something immoral).
                \item Emotions help to motivate us in doing the right thing.
            \end{itemize}
        \end{remark}
\end{description}

\begin{description}
    \item[Virtue ethics problems] \phantom{} 
        \begin{itemize}
            \item Virtue ethics does not provide rules to deal with conflicts between virtues.
            \item Virtue ethics might be too demanding depending on the standard set by the virtuous people.
            \item There is no absolute rule to select which virtuous people to consider. Depending on many factors, different people might choose different role models.
            \item There is no way to deal with disagreeing virtuous people.
            \item Virtue ethics considers actions right if done by virtuous people (and not the contrary, i.e., people are virtuous if they perform right actions). This boils down to the same problem of religion described in \Cref{sec:morality_other_systems}.
        \end{itemize}
\end{description}



\section{Principlism}

\begin{description}
    \item[Principlism] \marginnote{Principlism}
        Ethical theory originated from bioethics. It divides morality into two categories:
        \begin{descriptionlist}
            \item[Common morality] \marginnote{Common morality}
                Moral norms shared by all individuals. It supports human rights and moral ideals such as charity and generosity.

                \begin{remark}
                    Common morality follows absolutism and derives from human experience.
                \end{remark}

            \item[Particular morality] \marginnote{Particular morality}
                Specific and content-rich norms that should not violate common morality.

                \begin{description}
                    \item[Professional morality]
                        Moral norms specific to a profession.

                    \item[Public policy]
                        Regulations and guidelines promulgated by institutions.
                \end{description}
        \end{descriptionlist}
\end{description}


\begin{description}
    \item[Principles] \marginnote{Principles}
        General guidelines for the formulation of rules. The moral principles are: 
        \begin{enumerate*}[label=(\roman*)]
            \item respect for autonomy, 
            \item non-maleficence, 
            \item beneficence, and 
            \item justice.
        \end{enumerate*}

    \item[Rules] \marginnote{Rules}
        Instantiation of principles that are more specific in content and scope.

        \begin{description}
            \item[Substantive rules] \marginnote{Substantive rules}
                Rules of truth telling, confidentiality, informed consent, \dots

            \item[Authority rules] \marginnote{Authority rules}
                Rules that establish who can make decisions and perform actions.

            \item[Procedural rules] \marginnote{Procedural rules}
                Rules that establish a procedure to follow. They are the last resort if substantive and authority rules are not suited.
        \end{description}
\end{description}