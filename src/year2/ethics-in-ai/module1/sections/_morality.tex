\chapter{Introduction to ethics}



\section{Morality}

\begin{description}
    \item[Morality] \marginnote{Morality}
        There is no widely agreed definition of morality. On a high level, it refers to norms to determine which actions are right and wrong.  
\end{description}


\subsection{Conventional and critical morality}

\begin{description}
    \item[Positive (conventional) morality] \marginnote{Positive morality}
        Rules and principles created by humans that are widely accepted in a culture/society.

        \begin{remark}
            Conventional morality can change between different societies.
        \end{remark}
        
        \begin{remark}
            In principle, the starting moral values of an individual are those of the society it was born into.
        \end{remark}

    \item[Critical morality] \marginnote{Critical morality}
        Moral standards that are independent of conventional morality and are correct in general. It does not come from social agreements or beliefs.

        \begin{remark}
            Popular moral views are not necessarily true. Critical morality can be used as a ground-truth to determine whether conventional morality is correct.
        \end{remark}
\end{description}


\subsection{Branches of moral philosophy}

\begin{description}
    \item[Value theory] \marginnote{Value theory}
        Field that studies values (i.e., source of goodness and badness). Some research questions are: ``what is the good life?'', ``what is happiness?'', \dots

    \item[Normative ethics] \marginnote{Normative ethics}
        Field that studies what is morally required and how one should act. Some research questions are: ``what makes right actions right?'', ``do the ends justify the means?'', \dots

    \item[Non-normative ethics] \phantom{}
        \begin{description}
            \item[Descriptive ethics] \marginnote{Descriptive ethics}
                Field that uses scientific techniques to study how people reason and act.

            \item[Meta-ethics] \marginnote{Meta-ethics}
                Field that analysis the language, concepts, and methods of reasoning in normative ethics. Some research questions are ``can ethical judgments be true or false?'', ``does morality correspond to facts in the world?'', \dots
        \end{description}
\end{description}


\subsection{Morality vs other normative systems}

\begin{description}
    \item[Laws] \marginnote{Morality vs laws} 
        Laws are not necessarily in line with what should be morally correct. Some legal actions are immoral (e.g., cheating) and some illegal actions are moral (e.g., criticizing a dictator).

    \item[Etiquette] \marginnote{Morality vs etiquette}
        Standards of etiquette and good manners are not necessarily moral (e.g., forks should be put to the left of the plate, but it is not immoral to put them to the right).

    \item[Self-interest] \marginnote{Morality vs self-interest} 
        Sometimes, morality requires to sacrifice our well-being and, vice versa, immoral acts can improve our life.

    \item[Tradition] \marginnote{Morality vs tradition} 
        Practices that have been consolidated through time are not necessarily moral.
    
    \item[Religion] \marginnote{Morality vs religion} 
        Religions are based on the fact that there is a higher authority that created the set of moral norms. However, it is not clear whether God commands something because it considers them moral or things are moral because God commanded them.
\end{description}

% \begin{description}
%     \item[Ethics] 
%         Determine what is morally required.

%     \item[Meta-ethics] 
%         Study nature, scope, and meaning of moral judgement.
% \end{description}


\subsection{Absolutism and relativism}

\begin{description}
    \item[Absolutism] \marginnote{Absolutism}
        There is a single true ethics.

    \item[Relativism] \marginnote{Relativism}
        Judgment is relative to particular frameworks and attitudes.
\end{description}


\section{Morality in bioethics}


\subsection{Common and particular morality}

\begin{description}
    \item[Common morality] \marginnote{Common morality}
        Moral norms shared by all individuals. It supports human rights and moral ideals such as charity and generosity.

        \begin{remark}
            Common morality follows absolutism and derives from human experience.
        \end{remark}

    \item[Particular morality] \marginnote{Particular morality}
        Specific and content-rich norms that should not violate common morality.

        \begin{description}
            \item[Professional morality] \marginnote{Professional morality}
                Moral norms specific to a profession.

            \item[Public policy] \marginnote{Public policy}
                Regulations and guidelines promulgated by institutions.
        \end{description}
\end{description}


\subsection{Basic moral norms}

\begin{description}
    \item[Principles] \marginnote{Principles}
        General guidelines for the formulation of rules. The moral principles are: 
        \begin{enumerate*}[label=(\roman*)]
            \item respect for autonomy, 
            \item non-maleficence, 
            \item beneficence, and 
            \item justice.
        \end{enumerate*}

    \item[Rules] \marginnote{Rules}
        Instantiation of principles that are more specific in content and scope.

        \begin{description}
            \item[Substantive rules] \marginnote{Substantive rules}
                Rules of truth telling, confidentiality, informed consent, \dots

            \item[Authority rules] \marginnote{Authority rules}
                Rules that establish who can make decisions and perform actions.

            \item[Procedural rules] \marginnote{Procedural rules}
                Rules that establish a procedure to follow. They are the last resort if substantive and authority rules are not suited.
        \end{description}
\end{description}


\subsection{Conflicting moral norms}

\begin{description}
    \item[Moral dilemma] \marginnote{Moral dilemma}
        Situation where there are two conflicting norms or norms that lead to mutually exclusive actions.

        \begin{remark}
            Conflicts between morality and self-interest are not moral but practical dilemmas.
        \end{remark}
    
    \item[Prima facie moral judgment] \marginnote{Prima facie moral judgment}
        An action is wrong unless there are other norms that outweigh it.

    % \item[Pro tanto moral judgment] \marginnote{Pro tanto moral judgment}
    %     An action is wrong unless there is a justification.
\end{description}



\section{Consequentialism}

\begin{description}
    \item[Optimific action] \marginnote{Optimific action}
        Action that produces the best overall results.

    \item[Consequentialism] \marginnote{Consequentialism}
        Family of theories that consider an action morally required if and only if it is optimific.

        \begin{remark}
            Consequentialism sees morality as an optimization problem.
        \end{remark}
\end{description}


\subsection{Act utilitarianism}

\begin{description}
    \item[Act utilitarianism] \marginnote{Act utilitarianism}
        Instance of consequentialism that considers well-being the only thing that is intrinsically valuable.

    \item[Principle of utility] \marginnote{Principle of utility}
        Moral standard of act utilitarianism that considers an action morally required if and only if it is, among all the possibilities, the one that improves the overall world's well-being the most.
\end{description}

\begin{remark}
    Strengths of act utilitarianism are:
    \begin{itemize}
        \item It is egalitarian and impartial as everyone's utility counts the same.
        \item It justifies some basic moral intuitions (e.g., slavery is, in general, not optimific)
        \item It has by design a way of dealing with moral conflicts (i.e., choose the action that maximized well-being).
        \item It is flexible as actions violating some moral rules can be performed if they increase the overall well-being.
        \item It considers every person and every animal part of the moral community.
    \end{itemize}
\end{remark}

\begin{remark}
    Problems of act utilitarianism are:
    \begin{itemize}
        \item It is too demanding on the individuals as it requires constant self-sacrifice.
        \item It does not provide a decision procedure or a way to assess decisions.
        \item It has no room for impartiality (i.e., a family member is as important as a stranger).
        \item If the majority of society is against a minority group, unjust actions against the minority increases the overall world's well-being.
    \end{itemize}
\end{remark}


\subsection{Rule utilitarianism}

\begin{description}
    \item[Optimific social rule] \marginnote{Optimific social rule}
        Rule based on the idea that in the hypothetical case (nearly) everyone in a society were to accept it, the results would be optimific. 

    \item[Rule utilitarianism] \marginnote{Rule utilitarianism}
        An action is morally right if it is required by an optimistic social rule.

\end{description}

\begin{remark}
    Rule utilitarianism allows some degree of partiality. Moreover, an optimific action itself is not necessarily morally required if it is against an optimific social rule (e.g., it might be an optimific action to torture a prisoner, however, torture is forbidden by optimific social rules as it is more beneficial in the long term).
\end{remark}