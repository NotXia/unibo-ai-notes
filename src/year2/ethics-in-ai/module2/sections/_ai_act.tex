\chapter{AI Act}


\section{Introduction}


\subsection{General principles}
\marginnote{General principles}

Regulate the development of AI systems based on the principles of:
\begin{itemize}
    \item Human agency and oversight,
    \item Technical robustness and safety,
    \item Privacy and data governance,
    \item Transparency,
    \item Diversity, non-discrimination, and fairness,
    \item Social and environmental well-being.
\end{itemize}


\subsection{Definitions}

\begin{description}
    \item[AI system] \marginnote{AI system}
        Machine-based system that is designed to operate with varying levels of autonomy and adaptability. Moreover, its output is inferred from the input data.

        \begin{remark}
            Rule-based systems are excluded.
        \end{remark}

    \item[General purpose AI] \marginnote{General purpose AI}
        AI system that exhibits significant generality and is able to perform a wide range of tasks.
\end{description}


\subsection{Scope}

The AI Act applies to:
\begin{itemize}
    \item Providers who put an AI system on the EU's market, independently of their location.
    \item Deployers of AI systems located within the EU.
    \item Providers and deployers in third countries if the output produced is used in the EU.
    \item Importers and distributors of AI systems.
    \item Product manufacturers who use AI systems in their products.
    \item Authorized representatives of providers.
    \item People affected by AI systems in the EU.
\end{itemize}

\begin{remark}
    The AI Act is excluded for the following areas:
    \begin{itemize}
        \item Military, defense, and national security,
        \item Scientific research and development activities,
        \item Pre-market development and testing, if done in protected environments,
        \item International law enforcement cooperation, if fundamental rights safeguards are in place.
    \end{itemize}
\end{remark}

\begin{remark}
    The AI Act is a compromise between a product safety approach (e.g., minimum safety requirements, standards, \dots) and a fundamental rights approach.
\end{remark}

\begin{remark}
    The AI Act does not introduce new individual rights.
\end{remark}



\section{Risk regulation}

\begin{description}
    \item[Risk]
        Combination of the probability of harm and the severity of that harm.
\end{description}


\subsection{Risk levels}

\begin{description}
    \item[Unacceptable-risk (article 5)] \marginnote{Unacceptable-risk}
        Includes AI systems that are used for:
        \begin{itemize}
            \item Deploying harmful and manipulative subliminal techniques (i.e., beyond individual cognition),
            \item Exploiting vulnerable groups,
            \item Social scoring,
            \item Real-time remote biometric identification in public spaces for law enforcement purposes (with some exceptions),
            \item Biometric categorization of protected features,
            \item Predicting criminal offenses solely based on profiling or personality traits,
            \item Creating facial recognition databases by scraping the Internet or CCTV footage,
            \item Inferring emotions in workplaces or educational institutions, unless for medical or safety reasons.
        \end{itemize}

    \item[High-risk (article 6)] \marginnote{High-risk}
        Includes the following groups of AI systems:
        \begin{itemize}
            \item Those used as safety components of a product or falling under the EU health and safety legislation (e.g., toys, aviation, cars, medical devices, \dots)
            \item Those used in the following specific areas: biometric identification, critical infrastructures, education, employment, access to essential services, law enforcement, migration, and juridical and democratic processes.
            \item Those that performs profiling of natural persons.
        \end{itemize}

        Requirements to assess the impact of a these systems are:
        \begin{itemize}
            \item Determining the categories of natural persons and groups affected by the system,
            \item Checking compliance with European and national laws,
            \item Fundamental rights risk assessment (FRIA),
                \begin{example}[FRAIA]
                    Questionnaire created by the Dutch government to assess the impact of AI systems on fundamental rights.
                \end{example}
            \item Determining the risk of harm towards vulnerable groups and the environmental impact,
            \item Determining a plan for risk mitigation,
            \item Creating a governance system for human oversight, complaint handling, and redress.
        \end{itemize}

        \begin{remark}
            These requirements are still under research.
        \end{remark}

    \item[Limited-risk (article 52)] \marginnote{Limited-risk}
        Involves AI systems that interact with users with limited effects. It includes chatbots, emotion recognition, deep fakes, \dots

        Requirements are:
        \begin{itemize}
            \item The user must be informed that it is interacting with an AI system,
            \item Artificial content must be labeled as generated and contain detectable watermarks,
            \item Employers must inform workers on whether AI is used in the workplace and the reasons.
        \end{itemize}
    
    \item[Minimal-risk (article 69)] \marginnote{Minimal-risk}
        Involves AI systems with low to no effects on the user. It includes spam filters, video games, purchase recommendation systems, \dots.
        
        They are required to comply with the existing regulation but are not further regulated by the AI Act.

        \begin{remark}
            Providers of these systems are nonetheless encouraged to voluntarily respect high-risk requirements.
        \end{remark}

    \item[General purpose AI requirements] \marginnote{General purpose AI requirements}
        Specific requirements for general purpose AI are:
        \begin{itemize}
            \item Technical documentation must be kept for training, testing, and performance,
            \item Key information must be shared with downstream AI system providers,
            \item A summary of the training data must be published,
            \item Copyright compliance must be declared,
            \item Collaboration with regulators,
            \item Codes of practice should be provided.
        \end{itemize}

        \begin{remark}
            There is a subgroup of general purpose AI systems that includes those that pose a systemic risk. Additional requirements for these systems are:
            \begin{itemize}
                \item Additional risk mitigation,
                \item Independent system evaluation and model registration.
            \end{itemize}
        \end{remark}

        \begin{remark}
            If the computing power to train a model exceeds a certain threshold, that system is presumed to be a general purpose AI that poses systemic risks.
        \end{remark}
\end{description}


\subsection{Enforcement}

\begin{description}
    \item[Enforcement] \marginnote{Enforcement} 
        National supervisory authority enforces the AI Act in each member state with the support of the European Artificial Intelligence Office.
\end{description}


\subsection{AI regulatory sandboxes}

\begin{description}
    \item[AI sandbox] \marginnote{AI sandbox}
        Voluntary framework organized by member states for small to medium companies to test AI systems in controlled environments.
\end{description}



\section{AI liability}

\begin{remark}
    In the case of AI systems, liability has to account for:
    \begin{itemize}
        \item Black-box models,
        \item Autonomous and unpredictable models,
        \item Multiple actors and diffused responsibility,
        \item Lack of a clear legal framework,
        \item Difficulty in finding the causal chain.
    \end{itemize}
\end{remark}


\subsection{Liability theories}

\begin{description}
    \item[Strict liability] \marginnote{Strict liability}
        The producer is always responsible for their product both if it is their fault or due to negligence. The injured party only has to prove that damage occurred.

    \item[Fault liability] \marginnote{Fault liability}
        The defender has to show that someone is responsible for causing damage intentionally or negligently.

    \item[Mandatory insurance] \marginnote{Mandatory insurance}
        Enforce that the product (e.g., AI system) is covered by an insurance.

    \item[Compensation funds] \marginnote{Compensation funds}
        Economic relief for the users in case of damage or bankruptcy of the company.
\end{description}


\subsection{Revised Product Liability Directive}

\begin{description}
    \item[Revised Product Liability Directive] \marginnote{Revised Product Liability Directive}
        Product Liability Directive extended to software and AI systems. It is applied in all member states (i.e., maximum harmonization) and is based on the strict liability theory.

        The requirements to prove for compensation are that:
        \begin{itemize}
            \item The product is defective,
            \item Damage was caused,
            \item There is a causal link between defect and damage.
        \end{itemize}

    \item[Product] \marginnote{Product}
        The revised Product Liability Directive extends the definition of product with:
        \begin{itemize}
            \item Software and its updates,
            \item Digital manufacturing files (e.g., model for 3D printers),
            \item Digital services.
        \end{itemize}

        \begin{remark}
            Free and non-commercial open-source software are excluded
        \end{remark}

    \item[Liable parties (article 8)] \marginnote{Liable parties}
        The revised Product Liability Directive extends liable entities with:
        \begin{itemize}
            \item Any economic operator that has substantially modified the product outside the control of the manufacturer,
            \item Distributors of defective products,
            \item Online platforms.
        \end{itemize}
    
        \begin{remark}
            In the case of AI systems, the producer is the provider defined in the AI Act.
        \end{remark}

    \item[Types of damage (article 6)] \marginnote{Types of damage}
        Compensation can be provided for:
        \begin{itemize}
            \item Death or personal injury, including psychological health.
            \item Damage or destruction of properties, with the exception of the product itself, other components the defective product is integrated with, and products used for professional purposes only.
            \item Destruction or corruption of data that is not used for professional purposes.
        \end{itemize}

    \item[Defectiveness (article 7)] \marginnote{Defectiveness}
        In the case of software, liability is applied also for defects that come out after the product has been put in the market. This includes:
        \begin{itemize}
            \item Software updates under the manufacturer's control,
            \item Failure to address cybersecurity vulnerabilities,
            \item Machine learning.
        \end{itemize}

    \item[Presumption of defectiveness and causality (article 10)] \marginnote{Presumption of defectiveness and causality}
        Defectiveness is presumed when:
        \begin{itemize}
            \item The manufacturer fails to comply with the obligation to disclose information,
            \item A product does not comply with mandatory safety requirements,
            \item Damage is caused by an obvious product malfunction.
        \end{itemize}

        A causal link is presumed when:
        \begin{itemize}
            \item The damage is consistent with the type of defect,
            \item The technical or scientific complexity makes it difficult to prove liability (e.g., as with black-box models).
        \end{itemize}
\end{description}

\begin{remark}
    The revised Product Liability Directive does not cover:
    \begin{itemize}
        \item Discrimination,
        \item Violation of privacy (that are not already covered in the GDPR),
        \item Use of AI for professional purposes,
        \item Sustainability effects.
    \end{itemize}
\end{remark}


\subsection{AI Liability Directive}

\begin{description}
    \item[AI Liability Directive] \marginnote{AI Liability Directive}
        Additional protection for cases not covered in the revised Product Liability Directive. It is based on the fault liability theory.

        The directive has been cancelled by the EU Commission.
\end{description}


\begin{example}[Case study: delivery robot accident]
    An autonomous delivery robot that is able to navigate the pavement of pedestrian areas falls on the edge and hits a bicycle courier on the cycle lane. Both the biker and the robot sustained injuries/damage.
    
    \begin{descriptionlist}
        \item[AI Act] The system falls under the high-risk level (autonomous vehicle).
        \item[Revised Product Liability Directive] \phantom{}
            \begin{itemize}
                \item Liability can be sought after the company deploying the robots or the one renting them.
                \item The defect is related to the sensors/decision-making of the robot.
                \item Injuries are both physical and possibly psychological.
            \end{itemize}
    \end{descriptionlist}
\end{example}

\begin{example}[Case study: smart bank]
    A bank stores its data in the storage service provided by another company. An update released by the bank causes the corruption and loss of financial data.

    An affected customer failed to make payments leading to penalties and decrease in credit score.    

    \begin{descriptionlist}
        \item[AI Act] The system does not involve AI.
        \item[Revised Product Liability Directive] \phantom{}
            \begin{itemize}
                \item Liability can be sought after the bank. 
                \item The defect is the loss of records due to the update. 
                \item Damages are psychological and economical.
            \end{itemize}
    \end{descriptionlist}
\end{example}

\begin{example}[Case study: AI friend]
    A mental health chatbot is developed to support young users. However, a flaw in the system causes the generation of inappropriate and harmful messages.

    In an affected user, this lead to depression, self-harm, declining school performance, and withdrawal from social activities.

    \begin{descriptionlist}
        \item[AI Act] The system might fall under the unacceptable-risk level (manipulation) or under the high-risk level (medical diagnosis).
        \item[Revised Product Liability Directive] \phantom{}
            \begin{itemize}
                \item Liability can be sought after the company deploying the system.
                \item The defect is the flaw of the chatbot.
                \item Damage is psychological and physical. It also involves the loss of opportunities.
            \end{itemize}
    \end{descriptionlist}
\end{example}
