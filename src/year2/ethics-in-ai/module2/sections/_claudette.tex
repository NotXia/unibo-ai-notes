\chapter{CLAUDETTE}

\begin{description}
    \item[CLAUDETTE] \marginnote{CLAUDETTE}
        Clause detector (CLAUDETTE) is a system to classify clauses in a terms of service or privacy policy as:
        \begin{itemize}
            \item \textsc{Clearly fair},
            \item \textsc{Potentially unfair},
            \item \textsc{Clearly unfair}.
        \end{itemize}

    \item[Unfair contractual term (directive 93/13 art 3.1)] \marginnote{Unfair contractual term}
        A contractual term, that was not individually negotiated, is considered unfair if it causes a significant unbalance in the parties' rights and obligations.
\end{description}


\section{Unfairness categories}

\begin{description}
    \item[Consent by using clause] \marginnote{Consent by using clause}
        A clause is classified as:
        \begin{itemize}
            \item \textsc{potentially unfair}, if it states that the consumer accepts the terms of service by simply using the service.
        \end{itemize}
         
    \item[Privacy included] \marginnote{Privacy included}
        A clause is classified as:
        \begin{itemize}
            \item \textsc{potentially unfair}, if it states that the consumer consents to the privacy policy by simply using the service.
        \end{itemize}

    \item[Unilateral change] \marginnote{Unilateral change}
        A clause is classified as:
        \begin{itemize}
            \item \textsc{potentially unfair}, if the provider can unilaterally modify the terms of service or the service.
        \end{itemize}
    
    \item[Unilateral termination] \marginnote{Unilateral termination}
        A clause is classified as:
        \begin{itemize}
            \item \textsc{potentially unfair}, if the provider has the right to suspend or terminate the service and the reasons are specified.
            \item \textsc{clearly unfair}, if the provider can suspend or terminate the service for any reason.
        \end{itemize}

    \item[Jurisdiction clause] \marginnote{Jurisdiction clause}
        A clause is classified as:
        \begin{itemize}
            \item \textsc{clearly fair}, if consumers have the right to raise disputes in their place of residence.
            \item \textsc{clearly unfair}, if it only allows judicial proceedings in a different city or country.
        \end{itemize}

    \item[Choice of law] \marginnote{Choice of law}
        A clause is classified as:
        \begin{itemize}
            \item \textsc{clearly fair}, if the law of the consumer's country of residence is applied in case of disputes. 
            \item \textsc{potentially unfair}, in any other case.
        \end{itemize}

    \item[Arbitration clause] \marginnote{Arbitration clause}
        A clause is classified as:
        \begin{itemize}
            \item \textsc{clearly fair}, if arbitration is optional before going to court.
            \item \textsc{clearly unfair}, if arbitration should take place in a coutry different from the consumer's residence or should be based on the arbiter's discretion (and not by law).
            \item \textsc{potentially unfair}, in any other case.
        \end{itemize}

    \item[Limitation of liability] \marginnote{Limitation of liability}
        A clause is classified as:
        \begin{itemize}
            \item \textsc{clearly fair}, if the provider may be liable.
            \item \textsc{potentially unfair}, if the provider is never liable unless obliged by law.
            \item \textsc{clearly unfair}, if the provider is never liable (intentional damage included).
        \end{itemize}

    \item[Content removal] \marginnote{Content removal}
        A clause is classified as:
        \begin{itemize}
            \item \textsc{potentially unfair}, if the provider can delete or modify the user's content and the reasons are specified.
            \item \textsc{clearly unfair}, if the provider can delete or modify the user's content for any reason and without notice.
        \end{itemize}
\end{description}



\section{Methodology}

\begin{description}
    \item[Training data]
        Manually annotated terms of service.

    \item[Tasks] Two tasks are solved:
        \begin{description}
            \item[Detection] Binary classification problem aimed at determining whether a sentence contains a potentially unfair clause.
            \item[Sentence classification] Classification problem of determining the category of the unfair clause.
        \end{description}

    \item[Experimental setup]
        Leave-one-out where one document is used as test set and the remaining as train ($\frac{4}{5}$) and validation ($\frac{1}{5}$) set.

    \item[Metrics] Precision, recall, F1.
\end{description}


\subsection{Base clause classifier}

Experimented methods were:
\begin{itemize}
    \item Bag-of-words,
    \item Tree kernels,
    \item CNN,
    \item SVM,
    \item \dots
\end{itemize}


\subsection{Background knowledge injection}

\begin{description}
    \item[Memory-augmented neural network] \marginnote{Memory-augmented neural network}
        Model that, given a query, retrieves some knowledge from the memory and combines them to produce the prediction.

        In CLAUDETTE, the knowledge base is composed of all the possible rationales for which a clause can be unfair. The workflow is the following:
        \begin{enumerate}
            \item The clause is used to query the knowledge base using a similarity score and the most relevant rationale is extracted.
            \item The rationale is combined with the query.
            \item Repeat the extraction step until the similarity score is too low.
            \item Make the prediction and provide the rationales used as explanation.
        \end{enumerate}
\end{description}

\begin{example}[Knowledge base for liability exclusion]
    Rationales are divided into six class of clauses:
    \begin{itemize}
        \item Kind of damage,
        \item Standard of care,
        \item Cause,
        \item Causal link,
        \item Liability theory,
        \item Compensation amount.
    \end{itemize}
\end{example}


\subsection{Multilingualism}

\begin{description}
    \item[Training data]
        Same terms of service of the original CLAUDETTE corpus selected according to the following criteria:
        \begin{itemize}
            \item The ToS is available in the target language,
            \item There is a correspondence in terms of version or publication date between the documents in the two languages,
            \item There are structure similarities between the documents in the two languages.
        \end{itemize}
\end{description}


\begin{description}
    \item[Approaches] Different strategies have been experimented with:
    \begin{description}
        \item[Novel corpus for target language] \marginnote{Novel corpus for target language}
            Retrain CLAUDETTE from scratch with newly annotated data in the target language.

        \item[Semi-automated creation of corpus through projection] \marginnote{Semi-automated creation of corpus through projection}
            Method that works as follows:
            \begin{enumerate}
                \item Use machine translation to translate the annotated English document in the target language while projecting the unfair clauses.
                \item Match the machine translated document with the original document in the target language and project the unfair clauses (through human annotation).
                \item Train CLAUDETTE from scratch.
            \end{enumerate}

        \item[Training set translation] \marginnote{Training set translation}
            Translate the original document to the target language and train CLAUDETTE from scratch.

            \begin{remark}
                This method does not require human annotation.
            \end{remark}

        \item[Machine translation of queries] \marginnote{Machine translation of queries}
            Method that works as follows:
            \begin{enumerate}
                \item Translate the document from the target language to English.
                \item Feed the translated document to CLAUDETTE.
                \item Translate the English document back to the target language.
            \end{enumerate}
            
            \begin{remark}
                This method does not require retraining.
            \end{remark}
    \end{description}
\end{description}



\section{CLAUDETTE and GDPR}


\begin{description}
    \item[CLAUDETTE for GDPR compliance] 
        To integrate CLAUDETTE as a tool to check GDPR compliance, three dimensions, each containing different categories (ranked with three levels of achievement), are checked:
        \begin{descriptionlist}
            \item[Comprehensiveness of information] \marginnote{Comprehensiveness of information}
                Whether the policy contains all the information required by articles 13 and 14 of the GDPR.

                Categories of this dimension comprises:
                \begin{itemize}
                    \item Contact information of the controller,
                    \item Contact information of the data protection officer,
                    \item Purpose and legal bases for processing,
                    \item Category of personal data processed,
                    \item \dots
                \end{itemize}
        
            \item[Substantive compliance] \marginnote{Substantive compliance}
                Whether the policy processes personal data complying with the GDPR.

                Categories of this dimension comprises:
                \begin{itemize}
                    \item Processing of sensitive data,
                    \item Processing of children's data,
                    \item Consent by using, take-or-leave,
                    \item Transfer to third parties or countries,
                    \item Policy change (e.g., if the data subject is notified),
                    \item Licensing data,
                    \item Advertising.
                \end{itemize}
        
            \item[Clarity of expression] \marginnote{Clarity of expression}
                Whether the policy is precise and understandable (i.e., transparent).

                Categories of this dimension comprises:
                \begin{itemize}
                    \item Conditional terms: the performance of an action is dependent on a variable trigger.
                    \begin{remark}
                        Typical language qualifiers to identify this category are: depending, as necessary, as appropriate, as needed, otherwise reasonably, sometimes, from time to time, \dots
                    \end{remark}
                    \begin{example}
                        ``\textit{We also may share your information if we believe, in our sole discretion, that such disclosure is \underline{necessary} \textnormal{\dots}}''
                    \end{example}
                    
                    \item Generalization: terms to abstract practices with an unclear context.
                    \begin{remark}
                        Typical language qualifiers to identify this category are: generally, mostly, widely, general, commonly, usually, normally, typically, largely, often, primarily, among other things, \dots
                    \end{remark}
                    \begin{example}
                        ``\textit{We \underline{typically} or \underline{generally} collect information \dots When you use an Application on a Device, we will collect and use information about you in \underline{generally} similar ways and for similar purposes as when you use the TripAdvisor website.}''
                    \end{example}

                    \item Modality: terms that ambiguously refer to the possibility of actions or events.
                    \begin{remark}
                        Typical language qualifiers to identify this category are: may, might, could, would, possible, possibly, \dots

                        Note that these qualifiers have two possible meanings: possibility and permission. This category only deals with possibility.
                    \end{remark}
                    \begin{example}
                        ``\textit{We \underline{may} use your personal data to develop new services.}''
                    \end{example}

                    \item Non-specific numeric quantifiers: terms that are ambiguous in terms of actual measure.
                    \begin{remark}
                        Typical language qualifiers to identify this category are: certain, numerous, some, most, many, various, including (but not limited to), variety, \dots
                    \end{remark}
                    \begin{example}
                        ``\textit{\textnormal{\dots}we may collect a \underline{variety} of information, \underline{including} your name, mailing address, phone number, email address, \dots}''
                    \end{example}
                \end{itemize}
        \end{descriptionlist}
\end{description}



\section{LLMs and privacy policies}

\begin{remark}
    The GDPR requires two competing properties for privacy policies:
    \begin{descriptionlist}
        \item[Comprehensiveness] The policy should contain all the relevant information.
        \item[Comprehensibility] The policy should be easily understandable.
    \end{descriptionlist}
\end{remark}


\begin{description}
    \item[Comprehensive policy from LLMs] 
        Formulate privacy policies for comprehensiveness and let LLMs extract the relevant information.

        A template for a comprehensive policy could include:
        \begin{itemize}
            \item Categories of personal data collected,
            \item Purpose each category of data is processed for,
            \item Legal basis for processing each category,
            \item Storage period or deletion criteria,
            \item Recipients or categories of recipients the data is shared with, their role, the purpose of sharing, and the legal basis.
        \end{itemize}
\end{description}

\begin{description}
    \item[Experimental setup]
        The following questions were defined to assess a privacy policy:
        \begin{enumerate}
            \item What data does the company process about me?
            \item For what purposes does the company use my email address?
            \item Who does the company share my geolocation with?
            \item What types of data are processed on the basis of consent, and for what purposes?
            \item What data does the company share with Facebook?
            \item Does the company share my data with insurers?
            \item What categories of data does the company collect about me automatically?
            \item How can I contact the company if I want to exercise my rights?
            \item How long does the company keep my delivery address?
        \end{enumerate}

        Three scenarios were considered:
        \begin{itemize}
            \item Human evaluation of the questions on existing privacy policies,
            \item LLMs to answer the questions on ideal mock policies (with human evaluation).
            \item LLMs to answer the questions on real policies (with human evaluation).
        \end{itemize}

        Results show that:
        \begin{itemize}
            \item LLMs have high performance on the mock policies.
            \item LLMs and humans struggle to answer the questions on real privacy policies.
        \end{itemize}
\end{description}