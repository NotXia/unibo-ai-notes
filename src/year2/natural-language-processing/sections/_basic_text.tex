\chapter{Basic text processing}


\begin{description}
    \item[Text normalization]
        Operations such as:
        \begin{description}
            \item[Tokenization] \marginnote{Tokenization}
            Split a sentence in tokens.

            \begin{remark}
                Depending on the approach, a token is not always a word.
            \end{remark}

            \item[Lemmatization/stemming] \marginnote{Lemmatization/stemming}
                Convert words to their canonical form.

                \begin{example}
                    $\{ \texttt{sang}, \texttt{sung}, \texttt{sings} \} \mapsto \texttt{sing}$
                \end{example}

            \item[Sentence segmentation] \marginnote{Sentence segmentation}
                Split a text in sentences.

                \begin{remark}
                    A period does not always signal the end of a sentence.
                \end{remark}
        \end{description}
\end{description}


\section{Regular expressions}

\begin{description}
    \item[Regular expression (regex)] \marginnote{Regular expression (regex)}
        Formal language to describe string patterns.
\end{description}


\subsection{Basic operators}

\begin{description}
    \item[Disjunction (brackets)]
        Match a single character between square brackets \texttt{[]}.

        \begin{example}
            \texttt{/[wW]oodchuck/} matches \texttt{Woodchuck} and \texttt{woodchuck}.
        \end{example}

    \item[Range]
        Match a single character from a range of characters or digits.

        \begin{example} \phantom{}\\[-1.5em]
            \begin{itemize}
                \item \texttt{/[A-Z]/} matches a single upper case letter.
                \item \texttt{/[a-z]/} matches a single lower case letter.
                \item \texttt{/[0-9]/} matches a single digit.
            \end{itemize}
        \end{example}

    \item[Negation]
        Match the negation of a pattern.

        \begin{example}
            \texttt{/[\textasciicircum A-Z]/} matches a single character that is not an upper case letter.
        \end{example}

    \item[Disjunction (pipe)]
        Disjunction of regular expressions separated by \texttt{|}.

        \begin{example}
            \texttt{/groundhog|woodchuck/} matches \texttt{groundhog} and \texttt{woodchuck}.
        \end{example}


    \item[Wildcards] \phantom{}
        \begin{description}
            \item[Optional]
                A character followed by \texttt{?} can be matched optionally.

                \begin{example}
                    \texttt{/woodchucks?/} matches \texttt{woodchuck} and \texttt{woodchucks}.
                \end{example}

            \item[Any]
                \texttt{.} matches any character.

            \item[Kleene \texttt{*}]
                A character followed by \texttt{*} can be matched zero or more times.

            \item[Kleene \texttt{+}]
                A character followed by \texttt{+} must be matched at least once.

            \item[Counting]
                A character followed by \texttt{\string{n,m\string}} must be matched from $n$ to $m$ times.

                \begin{example} \phantom{}\\[-1.5em]
                    \begin{itemize}
                        \item \texttt{\string{n\string}} matches exactly $n$ instances of the previous character.

                        \item \texttt{\string{n,m\string}} matches from $n$ to $m$ instances of the previous character.

                        \item \texttt{\string{n,\string}} matches at least $n$ instances of the previous character.
                        
                        \item \texttt{\string{,m\string}} matches at most $m$ instances of the previous character.
                    \end{itemize}
                \end{example}
        \end{description}

    \item[Anchors] \phantom{}
        \begin{description}
            \item[Start of line]
                \texttt{\textasciicircum} matches only at the start of line.

                \begin{example}
                    \texttt{/\textasciicircum a/} matches \texttt{\underline{a}} but not \texttt{ba}.
                \end{example}

            \item[End of line] 
                \texttt{\$} matches only at the end of line.

                \begin{example}
                    \texttt{/a\$/} matches \texttt{\underline{a}} but not \texttt{ab}.
                \end{example}

            \item[Word boundary]
                \texttt{\char`\\ b} matches a word boundary character.

            \item[Word non-boundary] 
                \texttt{\char`\\ B} matches a word non-boundary character.
        \end{description}

    \item[Aliases] \phantom{}
        \begin{itemize}
            \item \texttt{\char`\\ d} matches a single digit (same as \texttt{[0-9]}).
            
            \item \texttt{\char`\\ D} matches a single non-digit (same as \texttt{[\textasciicircum\char`\\ d]}).
            
            \item \texttt{\char`\\ w} matches a single alphanumeric or underscore character (same as \texttt{[a-zA-Z0-9\_]}).
            
            \item \texttt{\char`\\ w} matches a single non-alphanumeric and non-underscore character (same as \texttt{[\textasciicircum\char`\\ w]}).

            \item \texttt{\char`\\ s} matches a single whitespace (space or tab).
            
            \item \texttt{\char`\\ S} matches a single non-whitespace.
        \end{itemize}


    \item[Capture group]
        Operator to refer to previously matched substrings.

        \begin{example}
            In the regex \texttt{/the (.*)er they were, the \char`\\ 1er they will be/}, \texttt{\char`\\ 1} should match the same content matched by \texttt{(.*)}.
        \end{example}
\end{description}



\section{Tokenization}

\begin{description}
    \item[Lemma] \marginnote{Lemma}
        Words with the same stem and roughly the same semantic meaning.
        \begin{example}
            \texttt{cat} and \texttt{cats} are the same lemma.
        \end{example}

    \item[Wordform] \marginnote{Wordform}
        Orthographic appearance of a word.
        \begin{example}
            \texttt{cat} and \texttt{cats} do not have the same wordform.
        \end{example}
    
    \item[Vocabulary] \marginnote{Vocabulary}
        Collection of text elements, each indexed by an integer.

        \begin{remark}
            To reduce the size of a vocabulary, words can be reduced to lemmas.
        \end{remark}

    \item[Type / Wordtype] \marginnote{Type / Wordtype}
        Element of a vocabulary (i.e., wordforms in the vocabulary).
    
    \item[Token] \marginnote{Token}
        Instance of a type in a text.

    \item[Genre] \marginnote{Genre}
        Topic of a text corpus (e.g., short social media comments, books, Wikipedia pages, \dots).
\end{description}

\begin{remark}[Herdan's law]
    Given a corpus with $N$ tokens, a vocabulary $V$ over that corpus roughly have size:
    \[ \left\vert V \right\vert = kN^\beta \]
    where the typical values are $10 \leq k \leq 100$ and $0.4 \leq \beta \leq 0.6$.
\end{remark}

\begin{description}
    \item[Stopwords] \marginnote{Stopwords}
        Frequent words that can be dropped.

        \begin{remark}
            If semantics is important, stopwords should be kept. LLMs keep stopwords.
        \end{remark}
\end{description}

\begin{remark}
    For speed, simple tokenizers use regex.
\end{remark}