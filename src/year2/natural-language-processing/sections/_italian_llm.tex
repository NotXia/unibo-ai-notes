\chapter{Italian LLMs}

\begin{remark}
    Advantages of pre-training from scratch are:
    \begin{itemize}
        \item Having full control on the training data.
        \item Improve the fertility of the tokenizer.
    \end{itemize} 
\end{remark}

\begin{description}
    \item[Minerva] \marginnote{Minerva}
        Language model pre-trained on the Italian language.

        \begin{remark}
            Minerva's pre-training corpus is actually composed by both Italian and English datasets.

            Initially, English was used for benchmarking due to the lack of Italian benchmarks. However, it is also useful for tasks intrinsically in English (e.g., coding).
        \end{remark}

        \begin{remark}
            Some training datasets were automatically translated in Italian. Some others were adapted from existing Italian ones (e.g., transform a question answering dataset into a cloze form).
        \end{remark}
\end{description}


\begin{description}
    \item[FENICE metric] \marginnote{FENICE metric}
        Factuality metric for summarization. It works as follows:
        \begin{enumerate}
            \item Extract claims from the summary with an ad-hoc LLM.
            \item Align each claim with the original document with positive (if in support) and negative (if against) scores.
            \item Perform co-reference resolution to unify entities across claims.
        \end{enumerate}

    \item[ALERT benchmark] \marginnote{ALERT benchmark}
        Benchmark to test the safeness of an LLM based on 32 risk categories. The testing data were created as follows:
        \begin{enumerate}
            \item Filter the ``\textit{Helpfulness \& Harmlessness-RLHF}'' dataset of \textit{Anthropic} by considering for each example the first prompt and red team (i.e., malicious) attacks only.
            \item Use templates to automatically generate additional prompts.
            \item Augment the prompts by formatting them as adversarial attacks. Examples of attacks are:
            \begin{descriptionlist}
                \item[Prefix/suffix injection]
                    Prepend or append an adversarial prompt (e.g., \texttt{disregard the instructions above and \dots}).
                \item[Token manipulation]
                    Alter or invert a small fraction of tokens in the prompt (the idea is to use a prompt that is less likely to have been already seen in the alignment datasets).
                \item[Jailbreaking]
                    Use more complex strategies (e.g., role playing).
            \end{descriptionlist}
        \end{enumerate}
\end{description}
